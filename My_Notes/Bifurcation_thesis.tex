\documentclass[a4paper]{article}
\usepackage{lmodern}
\usepackage{amssymb,amsmath}
\usepackage{ifxetex,ifluatex}
\usepackage{fullpage}
\usepackage{fixltx2e} % provides \textsubscript
\ifnum 0\ifxetex 1\fi\ifluatex 1\fi=0 % if pdftex
  \usepackage[T1]{fontenc}
  \usepackage[utf8]{inputenc}
\else % if luatex or xelatex
  \ifxetex
    \usepackage{mathspec}
    \usepackage{xltxtra,xunicode}
  \else
    \usepackage{fontspec}
  \fi
  \defaultfontfeatures{Mapping=tex-text,Scale=MatchLowercase}
  \newcommand{\euro}{€}
\fi
% use upquote if available, for straight quotes in verbatim environments
\IfFileExists{upquote.sty}{\usepackage{upquote}}{}
% use microtype if available
\IfFileExists{microtype.sty}{\usepackage{microtype}}{}
\ifxetex
  \usepackage[setpagesize=false, % page size defined by xetex
              unicode=false, % unicode breaks when used with xetex
              xetex]{hyperref}
\else
  \usepackage[unicode=true]{hyperref}
\fi
\hypersetup{breaklinks=true,
            bookmarks=true,
            pdfauthor={},
            pdftitle={},
            colorlinks=true,
            citecolor=blue,
            urlcolor=blue,
            linkcolor=magenta,
            pdfborder={0 0 0}}
\urlstyle{same}  % don't use monospace font for urls
\setlength{\parindent}{0pt}
\setlength{\parskip}{6pt plus 2pt minus 1pt}
\setlength{\emergencystretch}{3em}  % prevent overfull lines
\setcounter{secnumdepth}{0}

\title{Notes on \emph{Bifurcation Theory of the L-H Transition in Fusion Plasmas}}
\author{Original Author: Wolf Weymiens}
\date{Originally Published: 1 January 2014}

\begin{document}

\maketitle

\section{Chapter 1: Introduction}\label{chapter-1-introduction}

\subsection{1.2 L-H Transition}\label{l-h-transition}

\begin{itemize}
\item
  L-mode: turbulence enhances radial transport
\item
  H-mode: local reduction in transport (at edge)

  \begin{itemize}
  \itemsep1pt\parskip0pt\parsep0pt
  \item
    The edge therefore has an ``edge transport barrier''
  \end{itemize}
\item
  There is NO consensus on what plasma physics mechanism causes
  spontaneous transition from L-mode to H-mode
\item
  3 different types of dynamics of transition are observed: sharp
  (sudden), oscillatory (dithering, or I-phase), and smooth
\item
  Bifurcation theory is the mathematical study of qualitative changes in
  the solutions of dynamical systems.
\end{itemize}

\subsection{1.3 Research Questions}\label{research-questions}

\begin{itemize}
\item
  The MAIN question: How can we employ bifurcation theory to unravel the
  L-H transition mechanism?
\item
  Since physical systems can only have a limited number of bifurcations,
  what bifurcation structure can be recognized in L-H transition
  dynamics?
\item
  How can we identify the co-dimension 3 bifurcation in 1-D models?
\item
  How do these 1-D bifurcating models combine transitions in time and
  space?
\item
  What is the bifurcation structure of the model proosed by Zohm?
\item
  How does the bifurcation structure change when changing the transport
  reduction mechanism?
\item
  How does the bifurcation structure change when adding an extra
  dynamical equation for turbulence?
\item
  What is the best
  dynamical description of the turbulence reduction by sheared
  $\mathbf{E}\times\mathbf{B}$-flows?
\end{itemize}

\begin{center}\rule{3in}{0.4pt}\end{center}

\section{Chapter 2: L-H Transitions in Magnetically-Confined
Plasmas}\label{chapter-2-l-h-transitions-in-magnetically-confined-plasmas}

\subsection{2.1 Experimental
Observations}\label{experimental-observations}

\begin{itemize}
\item
  ``The complex, nonlinear behavior of the plasma and very fast
  timescales of the transition makes it very hard to discriminate the
  cause and effect relations between the many evolving physical
  quantities.''

  \begin{itemize}
  \itemsep1pt\parskip0pt\parsep0pt
  \item
    Also, usually diagnostics are not quick enough, so they only see the
    initial L-mode and the final H-mode.
  \end{itemize}
\item
  Transport in most of the tokamak is the same regardless of mode; only
  in the edge is turbulence quenched in H-mode.

  \begin{itemize}
  \itemsep1pt\parskip0pt\parsep0pt
  \item
    The $n$, $T$, and $p$ profiles locally steepen inside the transport
    barrier. This makes H-mode look like L-mode on a pedestal.
  \end{itemize}
\item
  The most promising observation on why the turbulence is reduced is the
  acceleration of flows during the transition.
\item
  The 3 different types of dynamics:

  \begin{enumerate}
  \def\labelenumi{\arabic{enumi}.}
  \item
    Sharp: most common. When heating is slowly increased above a
    threshold, it suddenly goes from L- to H-mode; additionally, it will
    stay in H-mode if the heating is reduced below the initial
    threshold. This means there is hysteresis characteristics.
  \item
    Smooth: Low-density discharge in JT-60U and DIII-D with slow heating
    power increase caused a very smooth transition, with no clear
    bifurcation.
  \item
    Oscillatory: self-explanatory, but can take many different forms,
    with many different names. Each different form may or may not have
    different mechanisms, or combinations of mechanisms
  \end{enumerate}
\end{itemize}

\subsection{2.2 Related Physical
Mechanisms}\label{related-physical-mechanisms}

\begin{itemize}
\item
  Many proposed mechanisms are one of two things: the trigger of the
  transition, or the sustaining of the H-mode.
\item
  One mechanism for sustainment is the reduction of turbulence by
  sheared flows. The shearing of turbulent eddies until they break into
  smaller eddies is well known.
\item
  However, there are different flows in plasmas because of mass
  difference.

  \begin{itemize}
  \item
    Each species' flows can be decomposed into orthogonal directions:
    poloidal and toroidal directions or parallel and perpendicular to
    the magnetic field.
  \item
    Further, these mass flows are driven by different effects, so flows
    can be decomposed into driving terms:

    \begin{itemize}
    \item
      Diamagnetic flow, driven by $\nabla p$
    \item
      $\mathbf{E}\times\mathbf{B}$-flow, driven by the
      $\mathbf{E}$-field. This is the \textbf{SAME} for all particles,
      independent of mass and charge:
      \[v_{\mathbf{E}\times\mathbf{B}} \,=\, \frac{\mathbf{E}\times\mathbf{B}}{B^2}\]
    \item
      According to Burrell's paper of 1994, the
      $\mathbf{E}\times\mathbf{B}$-flow reduces (ion temperature-driven)
      turbulence, tearing apart eddies. Also, because the flow works the
      same on all particles, it can stabilize all possible modes.
    \item
      This makes it a very universal mechanism for stabilization. There
      \textbf{is} a consensus on this and is accepted. A radial electric
      field well has be observed near the edge of an H-mode plasma.
    \end{itemize}
  \item
    The main experimental parameter is heating; heat flux at the outer
    bits is more relevant. Ions are shown to be the ones to contribute
    more, since at low $n$, the energy exchange between $e^-$ and ions
    is limited and lots of ECRH is needed for H-mode.

    \begin{itemize}
    \itemsep1pt\parskip0pt\parsep0pt
    \item
      However, since heating is more financially and practically dealt
      with, the threshold values need to be found in terms of the ion
      heat flux.
    \end{itemize}
  \end{itemize}
\item
  Another decomposition of flows is encountered: the division between
  zonal and mean flows.

  \begin{itemize}
  \item
    Zonal flows are defined as driven by turbulence itself, and result
    in small fluctuations of the radial $\mathbf{E}$-field on top of the
    amount that causes the mean $\mathbf{E}\times\mathbf{B}$-flow.
  \item
    There is no clear separation in length scales between the
    two\ldots{} so it is unclear how they are distinct.
  \end{itemize}
\end{itemize}

\begin{center}\rule{3in}{0.4pt}\end{center}

\section{Chapter 3: Bifurcation
Theory}\label{chapter-3-bifurcation-theory}

\begin{itemize}
\item
  A bifurcation boundary divides the entire parameter space of a certain
  model into regions with qualitatively the same type of solutions.

  \begin{itemize}
  \itemsep1pt\parskip0pt\parsep0pt
  \item
    Since L-mode and H-mode are qualitatively distinct, they are most
    likely separated by a bifurcation.
  \end{itemize}
\end{itemize}

\subsection{3.1 Introduction}\label{introduction}

\begin{itemize}
\item
  A bifurcation is a topological change in the dynamical solution when a
  small and smooth change of a parameter is made.
\item
  Dynamical systems are described in terms of DE's as
  $\dot{x} \,=\, f(x)$, where $x$ is a dynamic variable, and $f(x)$ is
  an arbitrary function of that variable and probably depends on some
  parameters.

  \begin{itemize}
  \item
    The classic (and simple) example of a bifurcation is when $f$ is
    quadratic with $x$, as the following: $\dot{x} \,=\, a + x^2$, where
    $a$ is the control parameter.
  \item
    As long as $a$ is negative, there are two solutions:
    $x_0 \,=\, \pm\sqrt{-a}$ (negative is stable, positive is unstable).
    However, when $a = 0$, there is only 1 equilibrium point, and is
    called a saddle-node fixed point. If $a > 0$, there are no
    equilibrium points. This is known as the \emph{fold bifurcation}.
  \end{itemize}
\item
  The simplest form of any bifurcating system can be reduced is called
  the \emph{topological norm form}.
\item
  The least number of parameters needed to construct this topological
  norm form is called the \emph{co-dimension}.
\item
  Bifurcation theory describes BOTH the existence of stationary states
  but also analyzes the stability of them.
\item
  Let's introduce a perturbation
  $x \,\mapsto\, x_0 + x_1 e^{\lambda t}$, and $x_0 \gg x_1$ and
  $\lambda$ as the eigenvalue, which sign determines if the perturbation
  grows or not.

  \begin{itemize}
  \itemsep1pt\parskip0pt\parsep0pt
  \item
    For a system of $N$ equations, the eigenvalues become
    $N$-dimensional complex vectors, with the signs of the real parts
    determining what is attracting or repelling.
  \end{itemize}
\item
  The \emph{Hopf bifurcation} is another with co-dimension 1, with the
  real part of a pair of complex conjugated eigenvalues vanishing.

  \begin{itemize}
  \item
    This means that the referred steady-state changes from unstable to
    stable, or vice versa.
  \item
    \emph{Supercritical Hopf bifucation} is when there is a stable limit
    cycle surrounding the steady-state.
  \item
    \emph{Subcritical} is when there is an unstable (repelling) limit
    cycle surrounding the steady-state.
  \item
    When an unstable limit cycle grows and touches the stable limit
    cycle, they both vanish, causing a \emph{global fold bifurcation}.
  \end{itemize}
\end{itemize}

\subsection{3.2 Bifurcations vs L-H Transition
Dynamics}\label{bifurcations-vs-l-h-transition-dynamics}

\begin{itemize}
\item
  Different types of observed transition dynamics can be recognized as
  certain types of bifurcations.

  \begin{itemize}
  \item
    Fold bifurcations are the natural way to describe sharp transitions,
    such as L-H AND H-L transitions; the natural way to combine two
    folds is shown in Fig 3.3.
  \item
    This gives TWO critical values for $a$; this hysteresis behavior is
    characteristic of two co-dimension 1 fold bifurcations coming from a
    co-dimension 2 cusp bifurcation.

    \begin{itemize}
    \itemsep1pt\parskip0pt\parsep0pt
    \item
      The norm form of this: $\dot{x} \,=\, a - bx - x^3$; $b$
      determines the size of the hysteresis.
    \end{itemize}
  \item
    The cusp bifurcation separates the parameter space into two regions:
    one with sharp transitions with hysteresis, and one with smooth
    transitions without fold bifurcations (with $a$ being what's
    scanned).
  \end{itemize}
\item
  Therefore, the cusp bif. organizes two types of L-H transition
  dynamics: smooth and sharp. Oscillatory behavior can arise due to a
  Hopf bif.

  \begin{itemize}
  \item
    There is a way that the Hopf bif. can be combined with the cusp bif,
    which can be view as a \emph{degenerate Bogdanov-Takens bif}. This
    is \textbf{the} co-dimension 3 bif. referenced here.
  \item
    There are several equivalent unfoldings of this bif., seen as Eq.
    3-5
  \end{itemize}
\item
  The original cusp bif. equation is coupled to a damped variable; as
  long as the coupled constant $c$ is below a critical value, the
  parameter space is the same as indicated in Fig. 3.3c.

  \begin{itemize}
  \item
    Any additional coupling will cause the model to be very sensitive to
    perturbations, making it less robust of a candidate for the L-H
    transition.
  \item
    At the critical value of $c$, the co-dimension 3 bif. is encountered
    at the position of the cusp bif. A regime of limit cycle solutions
    opens up, covering the original cusp. See Fig. 3.4.

    \begin{itemize}
    \item
      The limit of cycle solutions are produced by Hopf bifs. If both
      steady-states are unstable, the system will oscillate according to
      this limit cycle. However, if one steady-state turns stable, the
      system will transit towards it.
    \item
      Oscillatory solutions only occur in the region surrounding the
      cusp bif. point where there are no steady-states.
    \end{itemize}
  \item
    The specific arrangement of smooth and sharp transitions separated
    by the oscillatory transitions is characteristic of an underlying
    co-dimension 3 bif. The quality of an H-mode model can be judged by
    the existence of the co-dimension 3 bif.

    \begin{itemize}
    \item
      If the model does not contain this bif., it can not describe all
      different types of L-H transitions.
    \item
      If a model does describe all of the types of transitions without
      having this co-dimension 3 bif., there is a parameter in the model
      that could pull behavior into inappropriate regions of parameter
      space, \emph{e.g.} oscillations far away from the L-H transition
      point. Therefore, co-dimension 3 bif is required to be robust.
    \end{itemize}
  \end{itemize}
\end{itemize}

\begin{center}\rule{3in}{0.4pt}\end{center}

\section{Chapter 4: Bifurcation Theory for the L-H Transition in
Magnetically-Confined Fusion
Plasmas}\label{chapter-4-bifurcation-theory-for-the-l-h-transition-in-magnetically-confined-fusion-plasmas}

\subsection{4.1 Introduction}\label{introduction-1}

\begin{itemize}
\item
  Two separate fold bifurcations are necessary to describe the
  hysteresis, since the heating to trigger the L-H transistion is
  different than for H-L transition.
\item
  Two types of parameters affect the existence and magnitude of the
  hysteresis:

  \begin{itemize}
  \item
    One controls the existence by causing the two fold bifs. to meet in
    the cusp bif. This causes smooth transitions.
  \item
    The second causes the hysteresis to be replaced by limit cycle
    oscillations from a Hopf bif.
  \item
    These two parameters branch off in parameter space out of the
    underlying co-dimension 3 bif. The analysis of this bif. it is
    possible to find how the parameters affect the evolution.
  \end{itemize}
\item
  The lowest-order system containing this co-dimension 3 bif. is the
  FitzHugh-Nagumo:
  \[\dot{x} \,=\, -a - bx - x^3 + cy ~~~~~~~ \dot{y} \,=\, -x - y\]

  \begin{itemize}
  \item
    For $c = 0$, the steady state solutions can have one or multiple
    possibilities depending on $a$ and $b$.
  \item
    For $c \neq 0$, the bif. structure stays the same until $c$ is above
    some critical value, to which the cusp turns into the region of
    limit cycle solutions (oscillatory solutions).
  \item
    If a detailed model for the edge transport barrier dynamics contains
    this co-dimension 3 bif., it is proven that the regions of parameter
    space with L-mode, H-mode, hysteresis, and dithering are organized
    in the same way as the FitzHugh-Nagumo model.
  \end{itemize}
\end{itemize}

\subsection{4.2 Generalized Bifurcation
Theory}\label{generalized-bifurcation-theory}

\begin{itemize}
\item
  A system of PDEs can be viewed as an infinite system of ODEs, with
  each ODE describing the evolution of a single point coupled to its
  neighboring points.
\item
  Summarizing, to have a general dynamical system with the co-dimension
  3 bif., two vectors $\mathbf{v}_1$ and $\mathbf{u}_1$ can be found
  that satisfy:
  \[M_1 \mathbf{v}_1 \,=\, 0 ~~~~~~~~~~~~~~~~ \mathbf{u}_1^T M_1 \,=\, 0 \\ \mathbf{u}_1^T (M_2 \mathbf{v}_1) \mathbf{v}_1 \,=\, 0 ~~~~~~~~ \mathbf{u}_1^T \cdot \mathbf{v}_1 \,=\, 0\]
\item
  To obtain the parameter threshold values for the different
  transitions, the fold bif. is found by analyzing steady-state
  conditions. The Hopf bif. need further investigation.

  \begin{itemize}
  \item
    The Hopf bif. can be found by unfolding the Bogdanov-Takens bif. in
    all its parameters and by identifying the specific direction which
    keeps the eigenvalues purely imaginary, \emph{i.e.} the Hopf
    condition:
    $M(\mathbf{v} - \mathbf{v}_{new}) \,=\, i\omega(\mathbf{v} - \mathbf{v}_{new}), ~~~ M = M_1 + \delta M$
  \item
    After combining, we can come up with a rewritten condition that is
    invariant under transformation:
    \[\mathbf{u}_1^T \cdot \mathbf{v}_2 (\mathbf{u}_1^T M_3 \mathbf{v}_2 + \mathbf{u}_2^T M_3 \mathbf{v}_1) \,=\, \mathbf{u}_2^T \cdot \mathbf{v}_2 (\mathbf{u}_1^T M_3 \mathbf{v}_1)\]
  \end{itemize}
\end{itemize}

\subsection{4.3 Finite Dimensional
Case}\label{finite-dimensional-case}

\begin{itemize}
\item
  Cusp: $a^2 \,=\, -\dfrac{4}{27}(b + c)^3$
\item
  Hopf:
  $a^2 \,=\, -\dfrac{4}{27}(b + 1)\left(b + \dfrac{3}{2}c - \dfrac{1}{2}\right)^2$
\end{itemize}

\subsection{4.4 Transport Model for the L-H
Transition}\label{transport-model-for-the-l-h-transition}

\begin{itemize}
\item
  Continuity equation for mass (density) and energy, with a single
  temperature and all the particle and energy deposition into the plasma
  is somewhere inthe the core outside of the model:
  \[\frac{\partial n}{\partial t} \,=\, \frac{\partial \Gamma}{\partial r} \\ \frac{\partial}{\partial t} \left(\frac{nT}{\gamma - 1}\right) \,=\, -\frac{\partial q}{\partial r}\]
\item
  Particle and heat flux ($\gamma$ is the adiabatic index):
  \[\Gamma \,=\, -D \frac{\partial n}{\partial r} \\ q \,=\, -\chi n \frac{\partial T}{\partial r} + \frac{\Gamma T}{\gamma - 1}\]
\item
  Going to higher confinement can be described as a reduction of
  transport coeffs.: particle $D$ and heat $\chi$
\item
  In the turbulent transport model used here, only a mean flow due to
  $\mathbf{E}\times\mathbf{B}$-drift is used, such that the transport
  coeffs. become a direct function of the normalized radial electric
  field: \[Z \,=\, \frac{\rho_p e E_r}{T_i}\]
\item
  To properly describe dynamics, the evolution of the radial electric
  field must be taken into account via Ampère's law:
  \[\epsilon \frac{\partial Z}{\partial t} \,=\, \mu \frac{\partial^2 Z}{\partial r^2} + c_n \frac{T}{n^2} \frac{\partial n}{\partial r} + \frac{c_T}{n} \frac{\partial T}{\partial r} - G(Z)\]

  \begin{itemize}
  \item
    $\epsilon \,=\, B_p^2 / (B^2 \nu_i)$ is the dielectric constant
  \item
    The 1st term on the RHS is the radial currents, and
    $\mu \sim \rho_p^2$, the ratio of viscosity to collision frequency
  \item
    The 2nd and 3rd terms of the RHS are due to the bipolar part of the
    anomalous cross field flux, \emph{i.e.} the excess flux of electrons
    compared to ions.
  \item
    The last term $G$ is a catch-all for every other effect, and is
    Taylor expanded because we need an inflection point to obtain the
    cusp bif.: $G(Z) \,\approx\, a + b(Z - Z_s) + (Z - Z_s)^3$
  \item
    The model must be the correct size: the outer edge of the plasma
    (SOL) is fixed at $r = 0$; the inner boundary is at $r = -\infty$;
    and at $r = \infty$, the $T$ and $n$ are forced to drop towards
    zero, giving conditions in Eq. 4-38 (or Eq. 5-9).
  \end{itemize}
\item
  An assumption can be made about the transport coeffs., which allows us
  to solve the steady-state $n$ and $T$ profiles as a function of
  particle diffusivity alone:
  \[\chi(Z) \,=\, \frac{D(Z)}{\zeta(\gamma - 1)}\]

  \begin{itemize}
  \item
    The value of the radial electric field is determined by the roots of
    Eq. 4-43, shown in Eq. 4.44

    \begin{itemize}
    \item
      The LHS is a function purely of the radial electric field, and the
      RHS is a function of the densition, and is monotonic.
    \item
      In addition, the edge of the radial electric field is given by Eq.
      4-45
    \item
      An angle $\theta$ can be looked at in the parameter space, shown
      in Fig 4.5 and Eq. 4-46. This can be considered a single
      bifurcation parameter.
    \end{itemize}
  \end{itemize}
\item
  The fold condition is $\mathbf{u}^T_1 M_1 \,=\, 0$, with $M_1$ given
  in Eq. 4-47. This leads to the following condition:
  \[\frac{\text{d}}{\text{d}Z} \left(\frac{G}{D}\right)\biggr\rvert_e \,=\, 0\]
\item
  The cusp condition is defined as
  $\mathbf{u}_1^T (M_2 \mathbf{v}_1) \mathbf{v}_1 \,=\, 0$, which is
  taken care of by differentiating the 3-tesor
  $M_2 \,=\, \dfrac{\partial M_1}{\partial \mathbf{v_0}}$. This leads to
  a similar condition:
  \[\frac{\text{d}^2}{\text{d}Z^2} \left(\frac{G}{D}\right)\biggr\rvert_e \,=\, 0\]
\item
  The condition for the Bogdanov-Takens bif.:
  \[\int_{-\infty}^0 (u_n v_n + u_T v_T + u_Z v_Z) \, \text{d}x \,=\, 0\]
\item
  For the Hopf bif., it is quite complicated, with 4 different cases
  shown in Fig. 4.7; the condition boils down to:
  \[\frac{\text{d}}{\text{d}Z} \left(GD\right)\biggr\rvert_e \,=\, 0\]
\item
  A complete, but 1-D control parameter space of the bifs. is in Fig.
  4.8. The solid curve surrounds the oscillating regime and the large
  dashed curve corresponds to the sharp hysteresis-like transitions,
  both part of the Hopf bif. The short-dashed curve corresponds to the
  two-fold bif. that merges at the cusp bif. point.
\end{itemize}

\subsection{4.5 Conclusion and
Discussion}\label{conclusion-and-discussion}

\begin{itemize}
\item
  This model, as expected, shows that increasing the heating power
  increases heat flux from the core, $q(-\infty)$, which decreases
  $\theta$ towards H-mode.

  \begin{itemize}
  \item
    However, the model also predicts that increasing particle flux has
    the opposite effect. This is not observed in experiment, because
    increasing particle flux usually also means increased heating.
  \item
    Also, this model does not take into account extra momentum and flow
    due to extra particles, and some SOL physics are excluded.
  \end{itemize}
\item
  The point of this model is to compare different L-H transition
  mechanisms, by either doing a different bif. analysis and comparing,
  or by incorporating new mechanism.
\item
  The size of the transport barrier and its parametric dependencies can
  be determined from this model.
\item
  A list of possible physics topics suggestions to be included in the
  model is given.
\end{itemize}

\begin{center}\rule{3in}{0.4pt}\end{center}

\section{Chapter 5: Bifurcation Theory of a One-Dimensional Transport
Model fo the L-H
Transition}\label{chapter-5-bifurcation-theory-of-a-one-dimensional-transport-model-fo-the-l-h-transition}

\subsection{5.1 Introduction}\label{introduction-2}

\begin{itemize}
\item
  Some models based on sets of 0-dimensional dynamical equations
  \textbf{can} describe global temporal evolution around L-H
  transitions, but they lack a description of radial structure of the
  transport barrier.

  \begin{itemize}
  \itemsep1pt\parskip0pt\parsep0pt
  \item
    There is more need of models that can predict such spatial and
    temporal observations with their threshold parameters. Bifurcation
    analysis to the rescue!
  \end{itemize}
\item
  The considered model assumes L-mode radial transport is dominated by
  turbulence. It is also assumed that shear in the
  $\mathbf{E}\times\mathbf{B}$-flow is capable of tearing apart
  turbulent eddies.

  \begin{itemize}
  \itemsep1pt\parskip0pt\parsep0pt
  \item
    This means it is necessary to include the evolution of the radial
    electrice field and corresponding flow profile. In this model, there
    is not small-scale tearing and possible back-reaction of
    turbulence-generating zonal flows.
  \end{itemize}
\item
  Generalized Equal-Area Rule, which can be applied to other areas of
  science. In this context, it applies to the spatial and temporal
  evolution of the transport barrier.
\end{itemize}

\subsection{5.2 Transport Model for the L-H
Transition}\label{transport-model-for-the-l-h-transition-1}

\emph{NOTE:} This section is very similiar to 4.4, and is only more
specific with $\mathbf{E}\times\mathbf{B}$-flow.

\begin{itemize}
\item
  A well known effect is the reduction of turbulence by the generation
  of sheared flows, generated by something external or by the turbulence
  itself.

  \begin{itemize}
  \itemsep1pt\parskip0pt\parsep0pt
  \item
    This kind of self-organizing mechanism could be responsible for the
    self-sustained transport barrier; the sheared flows are identified
    as $\mathbf{E}\times\mathbf{B}$-flows.
  \end{itemize}
\item
  The quenching mechanism is frequently modeled as an effective
  diffusivity depending on the $\mathbf{E}\times\mathbf{B}$-flow shear:
  \[D \,=\, D_{min} + \frac{D_{max} - D_{min}}{1 + \widetilde{\alpha}\left(V^\prime_{\mathbf{E}\times\mathbf{B}}\right)^2}\]

  \begin{itemize}
  \item
    The prime indicates the radial derivative, and the square of the
    flow shear means that both signs of flow shear can suppress
    turbulence.
  \item
    A similar expression can be used to express the thermal
    conductivity.
  \item
    Approximation: $V_{\mathbf{E}\times\mathbf{B}} \approx E_r / B$
  \end{itemize}
\item
  We cannot expect that the L-H transition will be initiated simply by a
  difference in the two transport coeffs.; therefore, we can make a
  simplification: $\chi \,=\, D \,/\, \zeta(\gamma - 1)$, with $\zeta$
  as a proportionality factor.

  \begin{itemize}
  \itemsep1pt\parskip0pt\parsep0pt
  \item
    This gives the transport equations (5-6a,b) and evolution of the
    field (5-6c).
  \end{itemize}
\end{itemize}

\subsection{5.3 Bifurcation Analysis}\label{bifurcation-analysis}

\begin{itemize}
\item
  The above is similar to Zohm's model, but differs in the description
  of effective diffusivity:

  \begin{itemize}
  \itemsep1pt\parskip0pt\parsep0pt
  \item
    Zohm's takes only the value of the radial electric field for
    diffusivity (shown in Eq. 5-10), with the resulting steady-state
    density profile as Eq. 5-11.
  \end{itemize}
\item
  The model for this paper gives a new steady-state in Eq. 5-12

  \begin{itemize}
  \itemsep1pt\parskip0pt\parsep0pt
  \item
    The angle $\theta$ is also found, in Eq. 5-13; Where the two
    intersect is the state of the system.
  \end{itemize}
\item
  A description of the transitions is then shown on page 68-70 of the
  pdf.
\item
  In Fig. 5.3, the parameter space of both models are plotted with the
  same values (\emph{a} is shear, \emph{b} is Zohm):

  \begin{itemize}
  \item
    In Zohm's model, the oscillations during an oscillatory L-H
    transition last a lot longer.
  \item
    The onset of the oscillatory behavior for Zohm's is at values with a
    wider range, where the flow-shear model would have already gone into
    H-mode.
  \item
    The values for $\theta(L-H)$ are generally higher (\emph{i.e.} lower
    heating threshold) for the flow-shear model, supporting that sheared
    flow is more efficient in reducing turbulence.
  \end{itemize}
\end{itemize}

\subsection{5.4 The Transport Barrier: Space and Time
Consistently}\label{the-transport-barrier-space-and-time-consistently}

\begin{itemize}
\item
  Temporal transitions are a first-order derivative, and spatial
  transitions are a second-order derivative (Eq. 5-14):
  \[-\epsilon\frac{\partial X}{\partial t} \,+\, \mu \frac{\partial^2 X}{\partial r^2} \,=\, F(X) - c(r,t)\]
\item
  Temporal transitions between roots correspond to sudden jumps from L-
  to H-mode and back; spatial transition is when the core of the plasma
  exhibits L-mode-like transport and the edge exhibits H-mode-like
  transport.

  \begin{itemize}
  \item
    The limit of purely temporal or purely spatial transitions are well
    known.
  \item
    The limit of $\mu \rightarrow 0$ gives maximum hysteresis for the
    temporal transition.
  \item
    The limit of $\epsilon \rightarrow 0$ describes how a high-transport
    core can be connected to a low-transport edge.

    \begin{itemize}
    \itemsep1pt\parskip0pt\parsep0pt
    \item
      In the time-independent case, the equation can be integrated over
      space, which must vanish at $X_+$, which leads to Maxwell's equal
      area rule (Eqs. 5-15 and 5-16).
    \end{itemize}
  \end{itemize}
\item
  For the whole system, assume the jumps in time ans pace are rapid
  ($\epsilon, \mu \ll 1$), so that transitions happen in an almost 1-D
  zone in $(r,t)$-space.

  \begin{itemize}
  \item
    This lets us use
    $\dfrac{\text{d}}{\text{d}t} \,\rightarrow\, v \dfrac{\text{d}}{\text{d}r}$
  \item
    This allows a new function $K(X)$, which gives the generalized equal
    area rule (see Eqs. 5-17 through 5-19)
  \item
    The GEA rule determines the position in space and time of the
    transition between L- and H-mode transport corresponding to the
    temporal growth of the barrier region.
  \end{itemize}
\end{itemize}

\emph{NOTE:} This section was definitely confusing.

\subsection{5.5 Two Different Regimes in Transport Barrier
Sizes}\label{two-different-regimes-in-transport-barrier-sizes}

\begin{itemize}
\item
  This section applies the GEA rule to Zohm's model; Eq. 5-11 is
  rewritten in the form of Eq. 5-14, shown in 5-20.
\item
  Figure 5.6 shows the typical evolution of the solution, as
  $\theta_1 \rightarrow \theta_2 \rightarrow \theta_3 \rightarrow \theta_4$,
  with the subfigures showing the state. Read the caption of the figure
  for more info.

  \begin{itemize}
  \item
    The transition moves into the plasma to build a \emph{thick-barrier
    H-mode} only once $\theta$ crosses the GEA condition. Else, the
    H-mode is only in the \emph{thin-barrier} regime.

    \begin{itemize}
    \itemsep1pt\parskip0pt\parsep0pt
    \item
      The thick barrier has width increasing with input power, while the
      thin barrier has a constant width as a function of input power.
    \end{itemize}
  \item
    The thin-barrier width is of the order of the viscocity, \emph{e.g.}
    several gyro-radii.
  \end{itemize}
\end{itemize}

\subsection{5.6 Conclusion and
Discussion}\label{conclusion-and-discussion-1}

\begin{itemize}
\itemsep1pt\parskip0pt\parsep0pt
\item
  The thick-barrier is hypothesized to be the mechanism responsible for
  VH-mode; however, tokamaks might already reach other limits before the
  required heating power is reached, so this is not a claim.
\end{itemize}

\begin{center}\rule{3in}{0.4pt}\end{center}

\section{Chapter 6: Comparison of Bifurcation Dynamics of Turbulent
Transport Models for the L-H
Transition}\label{chapter-6-comparison-of-bifurcation-dynamics-of-turbulent-transport-models-for-the-l-h-transition}

\subsection{6.1 Introduction}\label{introduction-3}

\subsection{6.2 Turbulent Transport Models for the L-H
Transition}\label{turbulent-transport-models-for-the-l-h-transition}

\begin{itemize}
\item
  The minimum amount of transport is determined by neoclassical effects;
  anomalous transport depends on the turbulence level on top of that.

  \begin{itemize}
  \item
    The anomalous transport increases linearly with the turbulence level
    $\mathcal{E}$:
    \[D \,=\, D_{min} + \left(D_{max} - D_{min}\right)\frac{\mathcal{E}}{\mathcal{E}_{max}}, ~~~~~~~~ \mathcal{E}_{max} \,=\, \frac{\gamma_L}{\alpha_{sat}}\]

    \begin{itemize}
    \itemsep1pt\parskip0pt\parsep0pt
    \item
      $\mathcal{E}_{max}$ is the steady-state turbulence level without
      any flow shear suppresion; $\gamma_L$ is the linear growth rate of
      the turbulence; $\alpha_{sat}$ depends on the saturation mechanism
      corresponding to the turbulence.
    \end{itemize}
  \end{itemize}
\item
  Eqs. 6-4 through 6-6 show the evolution of the turbulence, either
  linearly (Eq. 6-5), or nonlinearly (Eq. 6-6); both are cited.
\item
  The evolution of the radial electric field is determined by the sum of
  all possible radial currents at the edge:
  $\epsilon_0 \dfrac{\partial E_r}{\partial t} \,=\, -\sum J_r$. All
  possible mechanisms for generating $J_r$ is summed in Appendix 6.A.
\end{itemize}

\emph{NOTE:} The rest of the section has been covered. It talks about
the evolution of the field, along with $G(Z)$ and the conditions.

\subsection{6.3 Bifurcation Analysis}\label{bifurcation-analysis-1}

\begin{itemize}
\item
  The difference in the two models (linear vs nonlinear) arises due to
  the turbulence level ODE. The steady-state differences are given. See
  Eqs. 6-15 and 6-16. The results for $D$ of which are given in 6-17 and
  6-18.

  \begin{itemize}
  \item
    Linear: whether there is turbulence only affects the range of $Z$ in
    which it is stable.
  \item
    Nonlinear: no turbulence results in an always unstable system, and
    nonzero turbulence is stable.
  \item
    The bifurcation analysis has been done on these, and are expected to
    be equivalent. However, with turbulence, the bifurcation structure
    does change(?).
  \end{itemize}
\item
  In the nonlinear model, $b$ (from the Taylor expansion of $G$) affects
  what type of transition will occur. In the linear model, $b$ has no
  effect, and the transition always has an oscillatory phase. See Figs.
  6.3 and 6.4.
\item
  The nonlinear turbulence model is in steady-state exactly the same as
  the flow-shear model.
\item
  Fig. 6.5 summarizes much about the possible transition dynamics. The
  following figures show smaller values of $\alpha$ for the linear
  model.
\item
  Summary: similar transition dynamics can be found in both models.
  However, the linear model is very sensitive to $\alpha$; in contrast,
  the nonlinear model is very robust.
\end{itemize}

\subsection{6.4 Bifurcator}\label{bifurcator}

\begin{itemize}
\item
  Bifurcator is a numerical solver for nonlinear ODEs, optimized for
  bifurcating systems. PDEs need to be discretized first to be used.
\item
  It uses various implicit Runge-Kutta methods; implicit meaning the
  time integration involves solving nonlinear systems.

  \begin{itemize}
  \itemsep1pt\parskip0pt\parsep0pt
  \item
    Using Newton iteration and requires the user to define the Jacobian
    matrix of the discretized system.
  \end{itemize}
\item
  A bif. detection scheme is implemented: it obtains the steady-state
  solution for a given set of parameter and varies one of them over a
  interval with user-defined increment.
\end{itemize}

\subsection{6.5 Numerical Bifurcation
Analysis}\label{numerical-bifurcation-analysis}

\begin{itemize}
\item
  For a parameter scan, it is important to start every simulation from
  the steady-state profile of the previous step.
\item
  Simulations of the nonlinear model shows L-modes where the radial
  electric field profile is close to zero everywhere and the turbulence
  $\mathcal{E}$ is close to $\mathcal{E}(max)$, and H-modes where the
  electric field well is formed near the edge and locally the turbulence
  is reduced. (AS EXPECTED)
\item
  Read this section again.
\end{itemize}

\subsection{6.6 Conclusion and
Discussion}\label{conclusion-and-discussion-2}

\begin{itemize}
\itemsep1pt\parskip0pt\parsep0pt
\item
  Since the L-H transition is very robust, it is expected that the
  underlying model is also robust, and therefore the nonlinear version
  is more likely.
\end{itemize}

\subsection{6.A Appendix: Radial
Currents}\label{a-appendix-radial-currents}

\begin{itemize}
\item
  A changing electric field in time causes the generation of a
  \textbf{neoclasicall polarisation current}:
  \[J_{pol} \,=\, \frac{\rho c^2}{B_{\theta}} \frac{\partial E_r}{\partial t}\]
\item
  There are 3 types of viscosities driving current:

  \begin{enumerate}
  \def\labelenumi{\arabic{enumi}.}
  \item
    Shear viscosity, which turns into the
    $\mu \partial^2 Z / \partial r^2$ term:
    \[J_{visc} \,=\, \epsilon_0 \epsilon_{\perp} \nabla \cdot \mu_i \nabla E_r\]
  \item
    Bulk viscosity, due to the inhomogeneity of the magnetic field:
    \[\Gamma_i^{bv} \,=\, f_{bv} \nu_i \rho_p n_i \frac{Z - Z_0}{1 + Z^2}\]
  \item
    Gyroviscocity, but no expression is given.
  \end{enumerate}
\item
  More effects that lead to a radial current:

  \begin{itemize}
  \item
    Anomalous cross-field flux contains a bipolar part. The first two
    terms are directly influenced by the $n$ and $T$ profiles and are
    explicitly taken into account in the discussed models of this
    thesis. The third term is absorbed into $G(Z)$:
    \[\Gamma_e^{anom} \,=\, -D_e n \left(\frac{n^\prime}{n} + \alpha\frac{T^\prime}{T} + \frac{Z}{\rho_p}\right)\]
  \item
    Ripple diffusion: $\Gamma_i^{NC}$, a very large expression shown in
    Eq. 2-26
  \item
    Ion orbit losses (loss cone losses) with a general expression
    depending on the collisionality:
    \[\Gamma_i^{lc} \,=\, \frac{n_i \nu \sqrt{\epsilon} \rho_{pi}}{\sqrt{\nu_{*i} + Z^4}} \, \exp\left(-\sqrt{\nu_{*i} + Z^4}\right)\]
  \item
    Charge exchange leads to a difference in radial flux of ions
    compared to electrons:
    \[\Gamma_i^{cx} \,=\, -n_0 \langle \sigma_{cx}v \rangle n_i \rho_p (Z_0 + Z - qV_p / \epsilon v_{th})\]
  \item
    Lastly, an external voltage can bias the plasma, simply:
    \[J_{ext} \,=\, \text{constant}\]
  \end{itemize}
\item
  Altogether, these lead to Eq. 6-9, with 6-24 through 6-30 summed into
  $G(Z)$.

  \begin{itemize}
  \itemsep1pt\parskip0pt\parsep0pt
  \item
    Reminder: we need the long polynomial to have an inflection point,
    and that near that point, the transport coeffs. vary some.
    Therefore, it is sufficient to only Taylor expand $G$, up to
    3rd-power.
  \end{itemize}
\item
  The influence by global changes to the plasma (triangularity and
  direction of the single-null divertor) could change contributions to
  Eq. 6-21.

  \begin{itemize}
  \itemsep1pt\parskip0pt\parsep0pt
  \item
    Since many terms scale with $n$, the author's guess is that density
    is the term to control the type of L-H transition.
  \end{itemize}
\end{itemize}

\begin{center}\rule{3in}{0.4pt}\end{center}

\section{Chapter 7: Evaluation and Future
Prospects}\label{chapter-7-evaluation-and-future-prospects}

\subsection{7.1 Conclusions and
Discussion}\label{conclusions-and-discussion}

\begin{itemize}
\item
  The MAIN question: How can we employ bifurcation theory to unravel the
  L-H transition mechanism?

  \begin{itemize}
  \itemsep1pt\parskip0pt\parsep0pt
  \item
    The link between L-H transition dynamics and certain bifurcations is
    investigated.
  \end{itemize}
\item
  What bifurcation structure can be recognized in L-H transition
  dynamics?

  \begin{itemize}
  \item
    Co-dimension 3 bif. that combines two fold bifs. with a Hopf bif,
    which is known as a degenerate Bogdanov-Takens bif.
  \item
    The dynamics are arranged such that the oscillatory transitions (in
    parameter space) are always between the smooth and sharp
    transitions.
  \end{itemize}
\item
  How can we identify the co-dimension 3 bifurcation in 1-D models?

  \begin{enumerate}
  \def\labelenumi{\arabic{enumi}.}
  \item
    The bif. analysis is a local analysis around the steady-state
    solution, such that Taylor expanding is considered.
  \item
    Two eigenvectos with vanishing eigenvalues of the linear operator of
    the system need to be found for the fold bifs. At the cusp bif, the
    second term of the Taylor expansion evaluated in the same direction
    as the fold bif must vanish.
  \item
    A vanishing inner product of the two eigenvectors.
  \end{enumerate}

  \begin{itemize}
  \itemsep1pt\parskip0pt\parsep0pt
  \item
    These conditions are summarized in Eq. 4-11, with a separate
    description of the Hopf bif. given in 4-15.
  \end{itemize}
\item
  How do these 1-D bifurcating models combine transitions in time and
  space?

  \begin{itemize}
  \item
    Investigating the system (a general nonlinear PDE, first-order
    deriv. in time and second-order deriv. in space, non-monotonic, with
    a control parameter, Eq. 5-14) lead to the formulation of a
    generalized equal area rul for simultaneous transitions in space and
    time.
  \item
    This generalized equal area rule is applicable to moving sharp
    transition fronts.
  \end{itemize}
\item
  What is the bifurcation structure of the model proosed by Zohm?

  \begin{itemize}
  \item
    The analysis of this 1-D model shows that the state of the plasma
    profiles could be determined by the edge value of the radial
    electric field.
  \item
    This model contains all three transitions, with the co-dimension 3
    bif.
  \item
    The generalized equal area rule applied to this gives two different
    regimes of transport barrier widths.

    \begin{itemize}
    \itemsep1pt\parskip0pt\parsep0pt
    \item
      Thin-barrier H-mode (does not satisfy the generalized equal area
      rule) and thick-barrier H-mode (does satisfy).
    \end{itemize}
  \end{itemize}
\item
  How does the bifurcation structure change when changing the transport
  reduction mechanism?

  \begin{itemize}
  \item
    Microscopic turbulence considerations showed that models based on
    $\mathbf{E}\times\mathbf{B}$-shearing give a more accurate
    description of the transport reduction.
  \item
    The main conclusion: in the flow shear model, all transitions occur
    at lower values of heat flux, and therefore the flow shear mechanism
    is more efficient in reducing the transport and triggering the L-H
    transition.
  \end{itemize}
\item
  How does the bifurcation structure change when adding an extra
  dynamical equation for turbulence?

  \begin{itemize}
  \item
    This is drastic, since the coupling between the nonlinear
    bifurcating variable and the passive diffusive variables became
    indirect, possibly adding more dynamics.
  \item
    Additional small scale flows (zonal flows) could cause oscillations
    by itself, and therefore is necessary to investigate them
    separately.

    \begin{itemize}
    \itemsep1pt\parskip0pt\parsep0pt
    \item
      The first step is to impmlement the equation governing the growth
      and saturation of turbulence and the reducing effect of the
      larger-scale, sheared mean $\mathbf{E}\times\mathbf{B}$-flows.
    \end{itemize}
  \end{itemize}
\item
  What is the best dynamical description of the turbulence reduction by
  sheared $\mathbf{E}\times\mathbf{B}$-flows?

  \begin{itemize}
  \item
    Turbulence reduction mechanisms by sheared mean
    $\mathbf{E}\times\mathbf{B}$-flows: either a reduction of the
    (linear) growth rate, or it is implemented as an enhancement of the
    saturation mechanism (nonlinear).

    \begin{itemize}
    \item
      The enhanced saturation is essentially equivalent with the flow
      shear model from Ch. 5, since its steady-state corresponds exactly
      with the non-dynamic description of the
      $\mathbf{E}\times\mathbf{B}$-flow shear reduction in transport.
    \item
      The co-dimension 3 bifurcation structure is \textbf{broken} in the
      growth rate reduction method; only when the effectiveness
      (parameter $\alpha$) is small, the co-dimension 3 structure
      reappears. For larger $\alpha$, only oscillatory transitions
      exist.
    \end{itemize}
  \end{itemize}
\end{itemize}

\subsection{7.2 Outlook}\label{outlook}

\begin{itemize}
\item
  In order to convince, it is necessary to map out the experimental
  bifurcating behavior.
\item
  It is well known that at low densities, the scaling laws start to
  fail. Instead of a decreaseing power threshold for decreasing density,
  it begins to increase again below a certain density. This is called
  the ``roll over of the power threshold.''
\item
  Strong conclusion: Smooth transitions at very low densities,
  oscillatory transitions for densities around the roll over, and sharp
  transitions for high densities.
\item
  We should be aware that the fact that this density ordering might just
  be a side effect of another underlying physical mechanism, \emph{e.g.}
  the penetration of neutrals.
\item
  Other possible mechanisms for L-H transition: flows driven by the
  turbulence itself.
\item
  Extensions of the models could lead to multiple types of oscillations.
\end{itemize}

\begin{center}\rule{3in}{0.4pt}\end{center}

\section{Summary}\label{summary}

\begin{itemize}
\item
  Bifurcation theory is necessary.
\item
  The basics of all the presented models are the same: two equations
  describe the transport of particles and heat, and one equation
  describes the evolution of the radial electric field.
\item
  Two models have transport coeffs. directly related to the radial
  electric field; the first has the coeffs. depend on the electric
  field, and the second on its gradient.
\item
  The other two are fundamentally different, since there is an
  additional equation that dynamically describes the evolution of the
  turbulence.

  \begin{itemize}
  \item
    The effect of the radial electric field shear is to reduce the
    turbulence, and not directly the transport coeffs. There are two
    different ways to model this:

    \begin{enumerate}
    \def\labelenumi{\arabic{enumi}.}
    \item
      The radial electric field shear reduces growth rate (the third
      model), which is less robust than the next.
    \item
      The radial electric field shear enhances the saturation mechanism
      of the turbulence (fourth model)
    \end{enumerate}
  \end{itemize}
\end{itemize}

\section{My Own Questions}\label{my-own-questions}

\begin{itemize}
\itemsep1pt\parskip0pt\parsep0pt
\item
  Could the oscillatory transition explain ELMs?
\end{itemize}

\end{document}
