\documentclass[]{article}
\usepackage{lmodern}
\usepackage{amssymb,amsmath}
\usepackage{ifxetex,ifluatex}
\usepackage{fixltx2e} % provides \textsubscript
\ifnum 0\ifxetex 1\fi\ifluatex 1\fi=0 % if pdftex
  \usepackage[T1]{fontenc}
  \usepackage[utf8]{inputenc}
\else % if luatex or xelatex
  \ifxetex
    \usepackage{mathspec}
  \else
    \usepackage{fontspec}
  \fi
  \defaultfontfeatures{Ligatures=TeX,Scale=MatchLowercase}
\fi
% use upquote if available, for straight quotes in verbatim environments
\IfFileExists{upquote.sty}{\usepackage{upquote}}{}
% use microtype if available
\IfFileExists{microtype.sty}{%
\usepackage[]{microtype}
\UseMicrotypeSet[protrusion]{basicmath} % disable protrusion for tt fonts
}{}
\PassOptionsToPackage{hyphens}{url} % url is loaded by hyperref
\usepackage[unicode=true]{hyperref}
\hypersetup{
            pdfborder={0 0 0},
            breaklinks=true}
\urlstyle{same}  % don't use monospace font for urls
\IfFileExists{parskip.sty}{%
\usepackage{parskip}
}{% else
\setlength{\parindent}{0pt}
\setlength{\parskip}{6pt plus 2pt minus 1pt}
}
\setlength{\emergencystretch}{3em}  % prevent overfull lines
\providecommand{\tightlist}{%
  \setlength{\itemsep}{0pt}\setlength{\parskip}{0pt}}
\setcounter{secnumdepth}{0}
% Redefines (sub)paragraphs to behave more like sections
\ifx\paragraph\undefined\else
\let\oldparagraph\paragraph
\renewcommand{\paragraph}[1]{\oldparagraph{#1}\mbox{}}
\fi
\ifx\subparagraph\undefined\else
\let\oldsubparagraph\subparagraph
\renewcommand{\subparagraph}[1]{\oldsubparagraph{#1}\mbox{}}
\fi

% set default figure placement to htbp
\makeatletter
\def\fps@figure{htbp}
\makeatother


\date{}

\begin{document}

\section{\texorpdfstring{Notes on \emph{Backstepping control of H-L
confinement transition in fusion plasma
model}}{Notes on Backstepping control of H-L confinement transition in fusion plasma model}}\label{notes-on-backstepping-control-of-h-l-confinement-transition-in-fusion-plasma-model}

\subsection{Chapter 2: Physics of the confinement transition and
H-mode}\label{chapter-2-physics-of-the-confinement-transition-and-h-mode}

\subsubsection{2.1 Order of scales}\label{order-of-scales}

\begin{itemize}
\item
  Transport of magnetic flux is the slowest process within a tokamak
  plasma; therefore, the magnetic field can be treated as a constant
  background field during transition.
\item
  Magnetic reconnection (\emph{e.g.} sawtooth instability) occurs on a
  fast time scale, but is not considered for this thesis.
\item
  Pressure gradients drive the turbulent modes, giving anomalous
  transport of particles, energy, and momentum.

  \begin{itemize}
  \tightlist
  \item
    Changes in turbulence occur in the smallest of time scales; they are
    often inferred from fluctuations in density and electric field.
  \end{itemize}
\end{itemize}

\subsubsection{2.2 Turbulence}\label{turbulence}

\paragraph{2.2.1 Dynamics}\label{dynamics}

\begin{itemize}
\item
  The dominant modes are the ion temperature gradient mode and trapped
  electron mode in a collisionless plasma.
\item
  Micro-instabilities give drift waves because electrons neutralize
  fluctuations in finite time.

  \begin{itemize}
  \tightlist
  \item
    The coupling of drift waves to other plasma modes yields different
    types of drift-wave turbulences such as the drift resistive
    ballooning mode and drift Alfvén turbulence.
  \end{itemize}
\item
  Drift waves are the only type of turbulence able to drive a radial
  flux and therefore, drift wave turbulence determines the level of edge
  turbulence and associated transport. Basic model for this:
  \[\frac{\text{d}\mathcal{E}}{\text{d}t} \,=\, \gamma_1 \mathcal{E} - \alpha_{sat}\mathcal{E}^2\]

  \begin{itemize}
  \tightlist
  \item
    \(\mathcal{E}\) is the turbulence level, \(\gamma_1\) is the growth
    rate, and \(\alpha_{sat}\) is the saturation rate.\_
  \end{itemize}
\end{itemize}

\paragraph{2.2.2 Flow shear suppression of
turbulence}\label{flow-shear-suppression-of-turbulence}

\begin{itemize}
\item
  A radial shear in plasma flow decorrelates turbulent eddies, which are
  small compared to the flow length scale. As eddies are extended by the
  sheared plasma flow, their wave energy is transferred to larger flow
  structures ad the expense of small turbulent structures:
  \[\frac{\partial V_{\mathbf{E}\times\mathbf{B}}}{\partial x} \,=\, \frac{\partial}{\partial x} \left(\frac{E_r}{B}\right) \,\approx\, \frac{1}{B_\phi} \frac{\partial E_r}{\partial x}\]
\item
  Both zonal and mean \(\mathbf{E}\times\mathbf{B}\) flow are involved
  in the transition.

  \begin{itemize}
  \item
    Zonal flows are produced by the shear of Reynolds stress arising
    from correlated radial and poloidal fluctuations.

    \begin{itemize}
    \tightlist
    \item
      Its shear suppresses the drift-wave turbulence, but the mechanism
      is damped by ion-ion collisions or flow instabilities. Therefore,
      zonal flows provide a temporary sink of energy, which show their
      signature before the transition.
    \end{itemize}
  \item
    The mean flow remains in H-mode, while the zonal flow vanishes.
    Shear flow suppression of turbulence has been modeled, in which
    sheared flow amplifies the saturation mechanism in H-mode, which is
    called the nonlinear suppression model:
    \[\frac{\text{d}\mathcal{E}}{\text{d}t} \,=\, \gamma_1 \mathcal{E} - \alpha_{sat} \left(1 + \alpha_{sup} \left(\frac{\partial Z}{\partial x}\right)\right) \mathcal{E}^2\]

    \begin{itemize}
    \tightlist
    \item
      The suppression rate coefficient is \(\alpha_{sup}\) and the
      normalized radial electric field is \(Z\). This nonlinear model is
      more robust compared to the linear one.
    \end{itemize}
  \end{itemize}
\end{itemize}

\subsubsection{2.3 Plasma momentum}\label{plasma-momentum}

\begin{itemize}
\item
  Internal drivers of momentum are the nonambipolar fluxes, while
  external momentum input is usually NBI.

  \begin{itemize}
  \tightlist
  \item
    Viscous stress damps the flow across flux surfaces, but also charge
    exchange friction with neutrals at the edge adds to damping.
  \end{itemize}
\item
  Momentum is a conserved variable that can be transported across flux
  surfaces to add a radial component to the plasma current. In
  steady-state, the radial ion force balance is:
  \[E_r \,=\, \frac{1}{n_i q_i} \nabla p_i + v_\phi B_\theta - v_\theta B_\phi\]

  \begin{itemize}
  \item
    Often, the radial field is determined experimentally by measure 3
    contributions:

    \begin{itemize}
    \item
      The poloidal and toroidal flow velocity can be measured using
      spectroscopy on radiating impurities assuming it represents the
      plasma flow velocity.
    \item
      The pressure gradient can be measured from density and temp
      measurements, but the error is amplified near the edge due to
      steep gradients.
    \end{itemize}
  \item
    Another approach to measure \(E_r\) requires the knowledge of the
    particle fluxes across magnetic flux surfaces. These fluxes can be
    classified as ambipolar and nonambipolar.

    \begin{itemize}
    \item
      Ambipolar fluxes are equal for electrons and ions, resulting in
      zero radial current.
    \item
      Nonambipolar \(\Gamma^{na}\) are different for various species and
      may violate the ambipolarity constraint:
      \[\langle \mathbf{J}\cdot \nabla \rho \rangle \,=\, \sum_s q_s \Gamma_s^{na} \,=\, 0\]
    \item
      As a result, \(E_r\) generates a return current such that the flux
      surface-averaged divergence of the plasma vanishes.
    \item
      Nonambipolar fluxes are also generated by kinetic and atomic
      processes.
    \end{itemize}
  \end{itemize}
\end{itemize}

\subsubsection{2.4 Nonambipolar particle
fluxes}\label{nonambipolar-particle-fluxes}

\begin{itemize}
\item
  The total nonambipolar particle flux \(\Gamma^{na}\) is split into
  contributions from volumetric \(\Gamma_V^{na}\) and edge
  \(\Gamma_E^{na}\) processes.
\item
  The volumetric components are summed up as:
  \[\Gamma_V^\text{na} \,=\, \underset{\substack{\text{bulk} \\ \text{viscosity}}}{\Gamma^{\pi \parallel}} + \underset{\substack{\text{shear} \\ \text{viscosity}}}{\Gamma^{\pi \perp}} + \underset{\text{polarization}}{\Gamma^\text{pol}} + \underset{\substack{\text{Reynolds} \\ \text{stress}}}{\Gamma^\text{Rey}} + \underset{\substack{\text{Maxwell} \\ \text{stress}}}{\Gamma^\text{Max}} + \underset{\substack{\text{resonant magn.} \\ \text{perturbations}}}{\Gamma^{\mathbf{J}\times\mathbf{B}}} + \underset{\substack{\text{poloidal flux} \\ \text{transient}}}{\Gamma^{\dot{\psi}_p}} + \underset{\text{external}}{\Gamma^\text{S}},\]

  \begin{itemize}
  \item
    \(\Gamma^{\pi\parallel, \pi \perp}\) is from viscous torque,
    \(\Gamma^\text{pol}\) is from plasma polarization,
    \(\Gamma^\text{Rey,Max}\) from fluctuation-induced Reynolds and
    Maxwell stress, \(\Gamma^{\mathbf{J}\times\mathbf{B}}\) is
    non-axisymmetric resonant magnetic perturbations,
    \(\Gamma^{\dot{\psi}_p}\) from momentum transport due to poloidal
    flux transients, and \(\Gamma^\text{S}\) as external momentums
    sources and sinks.\_

    \begin{itemize}
    \item
      {Gyroviscosity (viscosity in the diamagnetic direction)} is NOT
      taken into account, since its particle flux is negligible in a
      tokamak.
    \item
      Also Reynolds and Maxwell stress are left out (fluctuation-induced
      stresses) because they consider average plasma state variables.
    \item
      The poloidal flux transient is neglected since the magnetic field
      is assumed quasi-stationary around the transition time.
    \end{itemize}
  \item
    The effect of NBI (axisymmetric) resonant magnetic perturbations or
    ERCH may be added to the equation from the external term.
  \end{itemize}
\item
  At the edge, nonambipolar processes can be active due to the
  transition of closed to open magnetic field lines, the finite number
  of coils, or atomic processes due to the presence of neutrals:
  \[\Gamma^\text{na}_E \,=\, \underset{\substack{\text{anomalous} \\ \text{diffusion}}}{\Gamma^\text{an}} + \underset{\substack{\text{orbit} \\ \text{loss}}}{\Gamma^\text{ol}} + \underset{\substack{\text{charge} \\ \text{exchange}}}{\Gamma^\text{cx}} + \underset{\substack{\text{ripple} \\ \text{loss}}}{\Gamma^\text{rl}}\]

  \begin{itemize}
  \item
    \(\Gamma^\text{an}\) is from anomalous cross-field diffusion (drift
    waves), \(\Gamma^\text{ol}\) is from direct orbit loss (ion
    vs.~electron Larmor radius), \(\Gamma^\text{cx}\) is from charge
    exchange friction with neutrals, and \(\Gamma^\text{rl}\) is from
    ripple losses due to magnetic field inhomogeneities.

    \begin{itemize}
    \tightlist
    \item
      {Ripple losses} are excluded because low collisionality is
      assumed.
    \end{itemize}
  \end{itemize}
\item
  This leads to a full expression, excluding the aforementioned fluxes:
  \[\Gamma^\text{na} \,=\, \Gamma^{\pi \parallel} + \Gamma^{\pi \perp} + \Gamma^\text{pol} + \Gamma^\text{an} + \Gamma^\text{ol} + \Gamma^\text{cx}\]

  \begin{itemize}
  \tightlist
  \item
    Using the ambipolarity condition, the dynamics of the radial
    electric field can be derived from this.
  \end{itemize}
\end{itemize}

\subsubsection{2.5 L-H transition model}\label{l-h-transition-model}

\[\frac{\partial n}{\partial t} \,=\, \frac{\partial}{\partial x} \left(D(\mathcal{E})\frac{\partial n}{\partial x}\right) \\ \frac{\partial U}{\partial t} \,=\, \frac{\partial}{\partial x}\left(\chi(\mathcal{E}) n \frac{\partial T}{\partial x} + \frac{D(\mathcal{E}) T}{\gamma - 1} \frac{\partial n}{\partial x}\right)\]

\begin{itemize}
\item
  Conservation of mass and energy is above, with particle and heat
  diffusivities \(D(\mathcal{E})\) and \(\chi(\mathcal{E})\),
  respectively, and the ratio of ion-electron heat capacity (adiabatic
  index) \(\gamma\).

  \begin{itemize}
  \item
    The particle and heat diffusivities are simplified due to the belief
    that different diffusivities do not alter the bif behavior
    qualitatively. This is the simplification:
    \[\chi \,=\, \frac{D}{\zeta(\gamma - 1)}\]
  \item
    The additional parameter \(\zeta\) adapts the ratio between \(\chi\)
    and \(D\). Also, it is assumed that particle diffusivity scales
    linearly with turbulence \(\mathcal{E}\), which includes
    neoclassical and turbulent transport:
    \[D(\mathcal{E}) \,=\, D_{min} + (D_{max} - D_{min}) \frac{\mathcal{E}}{\mathcal{E}_{max}}\]

    \begin{itemize}
    \tightlist
    \item
      \(\mathcal{E}_{max} = \gamma_1 \alpha_{sat}^{-1}\) is the max
      turbulence level.
    \end{itemize}
  \item
    Plasma energy is rephrased: \(U \,=\, \dfrac{n T}{\gamma - 1}\).
    This simplifies the expression of {(2.11)} which can then be
    subtracted to give an expression of only plasma temperature:
    \[\frac{\partial T}{\partial t} \,=\, \frac{\partial}{\partial x} \left(\frac{D(\mathcal{E}) n}{\zeta} \frac{\partial T}{\partial x}\right) + \left(\frac{\zeta + 1}{\zeta} \frac{D(\mathcal{E})}{n} \frac{\partial n}{\partial x} \frac{\partial T}{\partial x}\right)\]
  \end{itemize}
\item
  The radial electric field PDE has radial currents from polarization
  and shear viscosity explicitly. The other nonambipolar processes are
  partly affected by the density and temperature gradients, which
  couples the differential equations.

  \begin{itemize}
  \item
    \(G(Z)\) is a cubic polynomial that approximates the remaining
    electric field-induces contributions, which has an inflection point
    in \(Z\)-space denoted as \(Z_S\):
    \[\epsilon \frac{\partial Z}{\partial t} \,=\, \mu \frac{\partial^2 Z}{\partial x^2} + \frac{c_n T}{n^2} \frac{\partial n}{\partial x} + \frac{c_T}{n} \frac{\partial T}{\partial x} + G(Z) ~~~~~~~~ (2.17)\]
  \item
    \(Z\) is the normalized electric field by normalizing \(E_r\) with
    respect to space (ion Larmor radius \(\rho_{pi}\)) and energy
    (temperature \(T\)).
    \[\text{NORMALIZATION:} ~~~~~ Z \,=\, \frac{\rho_{pi} e E_r}{T} ~~~~\text{with}~~~~ \rho_{pi} \,=\, \frac{m_i v_{ti}}{e B_\theta}\]
  \item
    The PDE of \(Z\) is not only nonlinear, but also includes a
    nonlinear coupling through temperature and density gradients.
  \item
    The boundary conditions are of the Robin-type at the separatrix for
    \(x = 0\):
    \[\frac{\partial n}{\partial x}\biggr\rvert_{x = 0} = \frac{n}{\lambda_n}, ~~~~~ \frac{\partial T}{\partial x}\biggr\rvert_{x = 0} = \frac{T}{\lambda_T}, ~~~~~ \frac{\partial Z}{\partial x}\biggr\rvert_{x = 0} = \frac{Z}{\lambda_Z}\]

    \begin{itemize}
    \item
      The SOL boundary conditions force the plasma variables to zero
      along a typical length scale \(\lambda\). The Neumann-type
      boundary conditions at the plasma core are given in terms of the
      particle and heat fluxes at \(x = L\), while the radial electric
      field shear vanishes:
      \[-\left[D(\mathcal{E}) \frac{\partial n}{\partial x}\right]_{x = L} = \Gamma_c, ~~~~~ -\left[\frac{D(\mathcal{E})}{\zeta} \frac{\partial T}{\partial x}\right]_{x = L} = \left[\frac{(\gamma - 1) q_c - T \Gamma_c}{n}\right]_{x = L}, ~~~~~ \left[\frac{\partial Z}{\partial x}\right]_{x = L} = 0\]

      \begin{itemize}
      \tightlist
      \item
        The (negative) fluxes flow towards the SOL, at \(x = 0\), for
        positive gradients that increase towards the plasma core, at
        \(x = L\).
      \end{itemize}
    \end{itemize}
  \item
    The model shows different transitions (smooth, oscillatory, or
    sharp) depending on the coefficients \(a\), \(b\), and \(c\) of
    \(G\): \[G(Z) \,=\, a + b(Z - Z_S) + c(Z - Z_S)^3\]
  \end{itemize}
\end{itemize}

\subsubsection{2.6 Radial electric field}\label{radial-electric-field}

\paragraph{2.6.1 Partial differential
equation}\label{partial-differential-equation}

\begin{itemize}
\item
  Ampère's law:
  \[\nabla \times \mathbf{B} \,=\, \mu_0 \mathbf{J} + \mu_0 \epsilon_0 \frac{\partial \mathbf{E}}{\partial t}\]

  \begin{itemize}
  \item
    The divergence in the radial direction is zero, meaning the plasma
    and displacement current must vanish. This means we obtain an
    equation for the radial electric field and radial current:
    \[\epsilon_0 \frac{\partial E_r}{\partial t} \,=\, -J_r \,=\, \sum_k \left(q_{e,i} \Gamma_{e,i}^k\right)\]

    \begin{itemize}
    \tightlist
    \item
      \(k\) represents some nonambipolar process, with \(q\,\Gamma\) as
      a particular radial current for electrons or ions (\(q\) is
      charge).
    \end{itemize}
  \item
    We can substitute this into the PDE, with some substitutions found
    in Appendix A:
    \[\epsilon_0 \frac{\partial E_r}{\partial t} \,=\, \underset{\text{bulk viscosity}}{e \Gamma_i^{\pi \parallel}} + \underset{\text{shear viscosity}}{\frac{\partial}{\partial x} \left(\frac{m_i n_i \mu_i}{B_\theta^2} \frac{\partial E_r}{\partial x}\right)} - \underset{\text{polarization}}{\frac{m_i n_i}{B^2} \frac{\partial E_r}{\partial t}} + \underset{\substack{\text{anomalous} \\ \text{diffusion}}}{e \Gamma_e^\text{an}} - \underset{\text{orbit loss}}{e \Gamma_i^\text{ol}} - \underset{\substack{\text{charge} \\ \text{exchange}}}{e \Gamma_i^\text{cx}}\]
  \item
    Rearrange, and add the polarization current to the displacement
    current:
    \[\epsilon_0 \left(1 + \frac{m_i n_i}{\epsilon B^2}\right) \frac{\partial E_r}{\partial t} \,=\, \frac{\partial}{\partial x} \left(\frac{m_i n_i \mu_i}{B_\theta^2} \frac{\partial E_r}{\partial x}\right) + e\left(-\Gamma_i^{\pi \parallel} + \Gamma_e^\text{an} - \Gamma_i^\text{ol} - \Gamma_i^\text{cx}\right)\]
  \item
    The pre-factor to the time derivative can be written in terms of the
    Alfvén velocity \(v_A\):
    \[\epsilon_0 \left(1 + \frac{m_i n_i}{\epsilon B^2}\right) \,=\, \epsilon_0 \left(1 + \frac{m_i n_i c_0^2 \mu_0}{\epsilon B^2}\right) \,=\, \epsilon_0 \left(1 + \frac{c_0^2}{v_A^2}\right), ~~~~~ v_A \,\approx\, \frac{B}{\sqrt{\mu_0 m_i n_i}}\]

    \begin{itemize}
    \tightlist
    \item
      It is also safe to assume that \(v_A \ll c_0\) at the plasma edge.
    \end{itemize}
  \item
    OVERALL, the PDE simplifies (assuming constant viscosity
    coefficient) to:
    \[m_i n_i \left(\frac{B_\theta}{B}\right)^2 \frac{\partial E_r}{\partial t} \,=\, m_i n_i \mu_i \frac{\partial^2 E_r}{\partial x^2} + e B_\theta^2 \left(-\Gamma_i^{\pi \parallel} + \Gamma_e^\text{an} - \Gamma_i^\text{ol} - \Gamma_i^\text{cx}\right)\]
  \end{itemize}
\end{itemize}

\paragraph{2.6.2 Normalization}\label{normalization}

\begin{itemize}
\item
  After a lengthy process, the model can be normalized:
  \[\frac{m_i n_i T_i}{e \rho_{pi}} \left(\frac{B_\theta}{B}\right)^2 \frac{\partial Z}{\partial t} \,=\, \frac{m_i n_i \mu_i T_i}{e \rho_{pi}} \frac{\partial^2 Z}{\partial x^2} + e B_\theta^2 \left(-\Gamma_i^{\pi \parallel} + \Gamma_e^\text{an} - \Gamma_i^\text{ol} - \Gamma_i^\text{cx}\right)\]

  \begin{itemize}
  \tightlist
  \item
    This model assumes the that the effects of
    \(\dfrac{\partial T_i}{\partial x}\) and
    \(\dfrac{\partial^2 T_i}{\partial x^2}\) on the shear viscosity are
    negligible.
  \end{itemize}
\end{itemize}

\paragraph{2.6.3 Coupling}\label{coupling}

\paragraph{2.6.4 Comparison}\label{comparison}

\begin{itemize}
\item
  The physics-based (as opposed to the `model'-based) electric field PDE
  can be put into the following form:
  \[\hat{\epsilon} \frac{\partial Z}{\partial t} \,=\, \hat{\mu} \frac{\partial^2 Z}{\partial x^2} + \frac{\hat{c}_n T}{n^2} \frac{\partial n}{\partial x} + \frac{\hat{c}_T}{n} \frac{\partial T}{\partial x} + \frac{T g(Z)}{n} ~~~~~~~~~~~ (2.38)\]

  \begin{itemize}
  \tightlist
  \item
    The electric field time scale is determined by \(\hat{\epsilon}\),
    while \(\hat{\mu}\) determines the layer thickness in which the
    electric field curvature is significant. These can thus be seen as
    time and `length' scales:
    \[\hat{\epsilon} \,=\, \frac{m_i T^2}{e \rho_{pi}} \left(\frac{B_\theta}{B}\right)^2, ~~~~~ \hat{\mu} \,=\, \frac{m_i \mu_i T^2}{e \rho_{pi}}\]
  \end{itemize}
\item
  The thermodynamic coefficients \(\hat{c}_n\) and \(\hat{c}_T\) can be
  expressed in terms of the nonambipolar contributions affected by the
  density and temperature gradient, respectively {(2.41)}.
\end{itemize}

\subsubsection{2.7 Hysteresis}\label{hysteresis}

\begin{itemize}
\item
  Hysteresis is often visualized by a gradient-flux relation,
  \emph{e.g.} electron pressure gradient versus separatrix heat flux:
  \(\nabla p_e(P_{sep})\).\_
\item
  Here, in the model of §2.5, the existence and size of the hysteresis
  are governed by the parameters \(b\) and \(c\).
\end{itemize}

\subsubsection{2.8 Edge transport barrier}\label{edge-transport-barrier}

\paragraph{2.8.1 H-mode physics}\label{h-mode-physics}

\begin{itemize}
\tightlist
\item
  The edge transport barrier is characterized by a pedestal height
  (maximum edge pressure) and width (radial extent of edge gradient).
  The pedestal width is considered an additional degree of freedom,
  determining the time-averaged pedestal pressure.
\end{itemize}

\paragraph{2.8.2 Control of edge transport
barrier}\label{control-of-edge-transport-barrier}

\subsubsection{2.9 Discussion}\label{discussion}

\begin{itemize}
\tightlist
\item
  Assumptions were made, most important being the basis of nonambipolar
  particle fluxes.
\end{itemize}

\subsection{Chapter 3: Model analysis}\label{chapter-3-model-analysis}

\subsection{Chapter 4: Boundary
control}\label{chapter-4-boundary-control}

\subsection{Chapter 5: Conclusion and
discussion}\label{chapter-5-conclusion-and-discussion}

\subsection{Appendix A: Nonambipolar particle
fluxes}\label{appendix-a-nonambipolar-particle-fluxes}

\subsubsection{A.1 Tokamak and plasma
parameters}\label{a.1-tokamak-and-plasma-parameters}

\subsubsection{A.2 Polarization current}\label{a.2-polarization-current}

\subsubsection{A.3 Ion (perpendicular) shear
viscosity}\label{a.3-ion-perpendicular-shear-viscosity}

\subsubsection{A.4 Ion (parallel) bulk
viscosity}\label{a.4-ion-parallel-bulk-viscosity}

\subsubsection{A.5 Residual Reynolds
stress}\label{a.5-residual-reynolds-stress}

\subsubsection{A.6 Electron anomalous
diffusion}\label{a.6-electron-anomalous-diffusion}

\subsubsection{A.7 Ion orbit loss}\label{a.7-ion-orbit-loss}

\subsubsection{A.8 Charge exchange
friction}\label{a.8-charge-exchange-friction}

\subsubsection{A.9 Ion ripple loss}\label{a.9-ion-ripple-loss}

\subsection{Appendix B: Linearization}\label{appendix-b-linearization}

\subsection{Appendix C: Spatial derivatives of
eigendecomposition}\label{appendix-c-spatial-derivatives-of-eigendecomposition}

\subsection{Appendix D: Characteristic ODE for the kernel
problem}\label{appendix-d-characteristic-ode-for-the-kernel-problem}

\subsection{Appendix E: First-order kernel
problem}\label{appendix-e-first-order-kernel-problem}

\subsection{Appendix F: Numerical implementation of kernel
problem}\label{appendix-f-numerical-implementation-of-kernel-problem}

\end{document}
