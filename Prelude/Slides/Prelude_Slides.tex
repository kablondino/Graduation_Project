\documentclass{beamer}
\title[$E_r$-Generating Terms in L-H Transitions]{Radial Electric Field-Generating Terms in L-H Transitions}
\author{Kevin A. Blondino \\
	Supervisor: dr. H.J. de Blank}
\institute[TUe]{Eindhoven University of Technology \\
	\medskip
	\textit{k.blondino@student.tue.nl}}
\date{14 August 2017}

%{{{ Presentation Style and Color
\mode<presentation>
{
%\usetheme{default}
%\usetheme{AnnArbor}
%\usetheme{Antibes}
%\usetheme{Bergen}
%\usetheme{Berkeley}
%\usetheme{Berlin}
%\usetheme{Boadilla}
%\usetheme{CambridgeUS}
%\usetheme{Copenhagen}
%\usetheme{Darmstadt}
%\usetheme{Dresden}
%\usetheme{Frankfurt}
%\usetheme{Goettingen}
%\usetheme{Hannover}
%\usetheme{Ilmenau}
%\usetheme{JuanLesPins}
%\usetheme{Luebeck}
%\usetheme{Madrid}
%\usetheme{Malmoe}
%\usetheme{Marburg}
%\usetheme{Montpellier}
%\usetheme{PaloAlto}
%\usetheme{Pittsburgh}
%\usetheme{Rochester}
%\usetheme{Singapore}
\usetheme{Szeged}
%\usetheme{Warsaw}

%\usecolortheme{albatross}
\usecolortheme{beaver}
%\usecolortheme{beetle}
%\usecolortheme{crane}
%\usecolortheme{dolphin}
%\usecolortheme{dove}
%\usecolortheme{fly}
%\usecolortheme{lily}
%\usecolortheme{orchid}
%\usecolortheme{rose}
%\usecolortheme{seagull}
%\usecolortheme{seahorse}
%\usecolortheme{whale}
%\usecolortheme{wolverine}

\setbeamertemplate{caption}[numbered]
\setbeamertemplate{bibliography item}{\insertbiblabel}
%\setbeamertemplate{footline} % To remove the footer line in all slides uncomment this line
%\setbeamertemplate{footline}[page number] % To replace the footer line in all slides with a simple slide count uncomment this line

%\setbeamertemplate{navigation symbols}{} % To remove the navigation symbols from the bottom of all slides uncomment this line
}
%}}}

\usepackage{amsmath,graphicx,booktabs}
\usepackage{caption}
\captionsetup{font=scriptsize,labelfont=bf}
\usepackage[UKenglish]{isodate}
\cleanlookdateon

%{{{ Setup References
\usepackage[backend=bibtex,url=false]{biblatex}
\addbibresource{../../References/References.bib}
%}}}

%{{{ Two Figures Side-by-Side
\newsavebox\IBoxA \newsavebox\IBoxB \newlength\IHeight
\newcommand\TwoFig[6]{% Image1 Caption1 Label1 Im2 Cap2 Lab2
	\sbox\IBoxA{\includegraphics[width=0.45\textwidth]{#1}}
	\sbox\IBoxB{\includegraphics[width=0.45\textwidth]{#4}}%
	\ifdim\ht\IBoxA>\ht\IBoxB
		\setlength\IHeight{\ht\IBoxB}%
	\else\setlength\IHeight{\ht\IBoxA}\fi
	\begin{figure}[ht]
		\minipage[t]{0.49\textwidth}\centering
			\includegraphics[height=\IHeight]{#1}
			\caption{#2}\label{#3}
		\endminipage\hfill
		\minipage[t]{0.49\textwidth}\centering
			\includegraphics[height=\IHeight]{#4}
			\caption{#5}\label{#6}
		\endminipage
	\end{figure}%
}
%}}}

\begin{document}
% Title page
\begin{frame}
\titlepage
\end{frame}

%--------------------------------------
% Overview
\begin{frame}
\frametitle{Overview}
\tableofcontents
\end{frame}

%--------------------------------------

\section{Background}
\subsection{H-mode}
\begin{frame}
\frametitle{H-mode}
	\begin{itemize}
		\item In 1982 at ASDEX, H-mode was discovered when heating was increased, which increased confinement time by a factor of around 2 \cite{wagner_development_1984}.
		\item Anomalous transport has been found to be the dominant transport of particles and energy.
		\item H-mode forms as a response to significant reduction in this anomalous transport at the plasma edge.
		\begin{itemize}
			\item A prevailing hypothesis is that high auxiliary power develops strong sheared plasma flow.
			\item It is characterized by its pressure profile that is significantly raised compared to L-mode, and is said to sit on a `pedestal.'
			\item Accordingly, there is a steep gradient at the edge, creating a transport barrier.
		\end{itemize}
	\end{itemize}
\end{frame}

\begin{frame}
\frametitle{H-mode}
	\begin{figure}
		\includegraphics[width=0.5\linewidth]{../../Graphics/L-mode_H-mode_compare.png}
		\caption{A comparison of the radial pressure profiles of L-mode and H-mode.
		The profile of H-mode can be thought of as on a ‘pedestal,’ in which the pressure profile is increase in the core.
		This is due to the transport barrier that is formed at the
		edge \cite{weymiens_bifurcation_2014}.}
		\label{fig:L-mode_H-mode_compare}
	\end{figure}
\end{frame}

%--------------------------------------

\subsection{Transition}
\begin{frame}
\frametitle{Transition}
	\begin{itemize}
		\item The L-H transition is a bifurcation in the turbulent transport at the plasma edge.
		\item The cusp bifurcation organizes two different transition dynamics: smooth and sharp.
		\begin{itemize}
			\item The smooth transition occurs for lower density plasmas.
			\item The sharp transition is more common, and exhibits hysteresis.
		\end{itemize}
		\item There is a third type of transition in which the plasma will oscillate between the two modes.
		\begin{itemize}
			\item It occurs when there are no stable states surrounding the cusp bifurcation.
			Mathematically, its existence is dicated by a coupling parameter above a critical value.
		\end{itemize}
	\end{itemize}
\end{frame}

\begin{frame}
\frametitle{Transition}
\TwoFig{../../Graphics/Bif_3D.png}
	{Two codimension 1 fold bifurcations, with the parameter $b$ dictating the size of the hysteresis, until the bifurcations merge into a cusp \cite{weymiens_bifurcation_2014}.}
	{fig:Bif_3D}
	{../../Graphics/3_transitions_single_simple.png}
	{Codimension 3 parameter space with the black line indicating the fold bifurcation. The parameter $b$ dictates the type of transition, including the size of the hysteresis in the sharp transition \cite{weymiens_bifurcation_2014}.}
	{fig:Bif_types}
\end{frame}

%--------------------------------------

\subsection{Electric Field}
\begin{frame}
\frametitle{Electric Field}
\end{frame}

%}}}
%--------------------------------------


\section{Transition Model}
\begin{frame}
\frametitle{Transition Model}
	The basic model for the transition, developed first by Itoh et al. \cite{itoh_edge_1991}, later expanded by Zohm \cite{zohm_dynamic_1994},  and then into the following form by Weymiens \cite{weymiens_bifurcation_2014}
It contains the continuity equations of energy and density to describe H-mode.
\begin{equation}
	\epsilon\,\frac{\partial Z}{\partial t} \,=\, \mu\,\frac{\partial^2 Z}{\partial x^2} + \frac{c_n T}{n^2} \frac{\partial n}{\partial x} + \frac{c_T}{n} \frac{\partial T}{\partial x} + G(Z)
	\label{eq:pde}
\end{equation}
This model uses $Z$, the radial electric field normalized with respect to the ion Larmor (gyro-)radius $\rho_{\theta i}$ and temperature $T = T_i = T_e$.
\begin{equation}
	Z \,=\, \frac{\rho_{\theta i} e E_r}{T}, ~~~~~ \rho_{\theta i} \,=\, \frac{m_i v_{\phi i}}{e B_\theta}
	\label{eq:normalization}
\end{equation}
\end{frame}

\begin{frame}
\frametitle{Transition Model}
\begin{equation}
	\epsilon\,\frac{\partial Z}{\partial t} \,=\, \mu\,\frac{\partial^2 Z}{\partial x^2} + \frac{c_n T}{n^2} \frac{\partial n}{\partial x} + \frac{c_T}{n} \frac{\partial T}{\partial x} + G(Z) \tag{\ref{eq:pde}}
	%\label{eq:pde}
\end{equation}
	\begin{itemize}
		\item The LHS represents the radial current due to polarization.
		\begin{itemize}
			\item $\epsilon \,\equiv\, \dfrac{B_\theta^2}{B^2 \nu_i}$ is the dielectric constant.
			It dictates the timescale, with smaller values corresponding to sharper temporal transitions.
		\end{itemize}
		\item The 2nd derivative term describes the radial current due to the anomalous shear viscosity of the $\mathbf{E}\times\mathbf{B}$ drift.
		\begin{itemize}
			\item $\mu$ is the ratio of viscosity to collision frequency. Smaller values result in sharper spatial transitions.
		\end{itemize}
	\end{itemize}
\end{frame}


\begin{frame}
\frametitle{Transition Model}
\begin{equation}
	\epsilon\,\frac{\partial Z}{\partial t} \,=\, \mu\,\frac{\partial^2 Z}{\partial x^2} + \frac{c_n T}{n^2} \frac{\partial n}{\partial x} + \frac{c_T}{n} \frac{\partial T}{\partial x} + G(Z) \tag{\ref{eq:pde}}
	%\label{eq:pde}
\end{equation}
	\begin{itemize}
		\item The 1st spatial derivatives are an accumulation of all 1st derivative terms from the remaining fluxes.
		\item $G(Z)$ is a cubic polynomial that approximates the remaining non-derivative terms of the fluxes
		\begin{equation}
			G(Z) \,=\, a + b(Z - Z_S) + c(Z - Z_S)^3
			\label{eq:G_func}
		\end{equation}
		\begin{itemize}
			\item coeffss
		\end{itemize}
	\end{itemize}
\end{frame}
%}}}
%--------------------------------------

\section{References}
\begin{frame}
\frametitle{References}
\renewcommand*{\bibfont}{\scriptsize}
%\nocite{*}
\printbibliography
\end{frame}
%}}}

\end{document}
