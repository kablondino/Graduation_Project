Fusion plasmas are inherently complicated systems, and thus are complicated to model. This is true, even when taking simplification liberties.
The model developed by Zohm, and later refined by others, assumes that the transport barrier occurs in a thin layer at the plasma edge.
The plasma edge is defined to be at the last-closed flux surface (separatrix).
This allows for slab geometry to be used, in which $\psi$ is the usual radial coordinate.
However, $x$ will be used as it is more distinct as a spatial coordinate.
The slab geometry reduces the model into one (spatial) dimensional differential equations.
One of the liberties taken to simplify the model is that a single temperature is used, \emph{i.e.} the electron and ion temperatures are identical.\todo{\color{red}{Liberties?}}

\section{Transport Equations}\label{sec:transport_eqs}
To model the transport of a fully-ionized fusion plasma, conservation of mass and energy are considered.
These lead to continuity equations of plasma density $n$ and internal energy $U$:
\begin{align} % n and U continuity
	\frac{\partial n}{\partial t} \,+\, \frac{\partial \Gamma}{\partial x} \,&=\, 0~,\label{eq:n_continuity} \\
	\frac{\partial U}{\partial t} \,+\, \frac{\partial q}{\partial x} \,&=\, 0\label{eq:U_continuity}
\end{align}
The particle flux $\Gamma$ is determined by some effective particle diffusion $D$ due to anomalous transport.
The heat flux $q$, however, is determined by effective heat convection.
This is the sum of heat diffusion $\chi n$, and advection $\Gamma T$. \todo{\color{red}{Check this statement}}
\begin{align} % Fluxes
	\Gamma \,&=\, -D \frac{\partial n}{\partial x}~,\label{eq:particle_flux} \\
	q \,&=\, -\chi n \frac{\partial T}{\partial x} \,+\, \frac{\Gamma T}{\gamma - 1} \label{eq:heat_flux}
\end{align}
The adiabatic index $\gamma$ will be set to $5/3$, as is customary for monatomic gases.
It is important to note that these fluxes omit explicit drift velocities as well as particle and heat sources from within the domain of the model \cite{zohm_dynamic_1994}.

The particle and heat diffusivities $D$ and $\chi$ are considered to be functions of the turbulence.
The L-H transition is not expected to be caused by a difference in form between the two diffusivities.
It is therefore assumed they are proportional.
\begin{align} % chi-D relation
	\chi \,=\, \frac{D}{\zeta (\gamma - 1)} \label{eq:heat_particle_diff_relation}
\end{align}
The parameter $\zeta$ determines the coupling strength of the two diffusivities. The specifics of the forms chosen for the diffusivities is discussed in Section~\ref{sec:diffusivities}.

In order to simplify the model, the following substitution to the internal energy can be made:
\begin{align} % U definition
	U \,\equiv\, \frac{n T}{\gamma - 1} \label{eq:U_definition}
\end{align}
Making this substitution, the substitution of the fluxes, and subsequently working out the product rules of the derivatives, much more compact forms of Equations~\ref{eq:n_continuity} and \ref{eq:U_continuity}.
One convenience is that Equation~\ref{eq:n_continuity} shows up as a collection of terms within Equation \ref{eq:U_continuity}.
The simplified forms of these transport equations is as such:
\begin{align}
	\frac{\partial n}{\partial t} \,&=\, \frac{\partial}{\partial x} \left[D \frac{\partial n}{\partial x}\right]~,\label{eq:n_compact} \\
	\frac{\partial T}{\partial t} \,&=\, \frac{\partial}{\partial x} \left[\frac{D}{\zeta} \frac{\partial T}{\partial x}\right] \,+\, \left(1 + \frac{1}{\zeta}\right) \frac{\partial n}{\partial x} \frac{\partial T}{\partial x} \label{eq:T_compact}
\end{align}

\section{Radial Electric Field}\label{sec:Z_equation}
Zohm's original form is as such:
\begin{align}
	\frac{\partial Z}{\partial t} \,&=\, -N(Z,g) + \mu \frac{\partial Z}{\partial x^2}
\end{align}

\section{Forms of Diffusivities}\label{sec:diffusivities}


