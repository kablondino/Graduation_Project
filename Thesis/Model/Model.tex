Fusion plasmas are inherently complicated systems, and thus are complicated to model. This is true, even when taking simplification liberties.
The model developed by Zohm, and later refined by others, assumes that the transport barrier occurs in a thin layer at the plasma edge \cite{zohm_dynamic_1994}.
The plasma edge is defined to be at the last-closed flux surface (separatrix).
This allows for slab geometry to be used, in which $\psi$ is the usual radial coordinate.
However, $x$ will be used as it is more distinct as a spatial coordinate.
The slab geometry reduces the model into one (spatial) dimensional differential equations.
One of the liberties taken to simplify the model is that a single temperature is used, \emph{i.e.} the electron and ion temperatures are identical.\todo{\color{red}{Liberties?}}

\section{Transport Equations}\label{sec:transport_eqs}
To model the transport of a fully-ionized fusion plasma, conservation of mass and energy are considered.
These lead to continuity equations of plasma density $n$ and internal energy $U$:
\begin{align} % n and U continuity
	\frac{\partial n}{\partial t} \,+\, \frac{\partial \Gamma}{\partial x} \,&=\, 0~,\label{eq:n_continuity} \\
	\frac{\partial U}{\partial t} \,+\, \frac{\partial q}{\partial x} \,&=\, 0\label{eq:U_continuity}
\end{align}
The particle flux $\Gamma$ is determined by some effective particle diffusion $D$ due to anomalous transport.
The heat flux $q$, however, is determined by effective heat convection.
This is the sum of heat diffusion $\chi n$, and advection $\Gamma T$. \todo{\color{red}{Check this statement}}
\begin{align} % Fluxes
	\Gamma \,&=\, -D \frac{\partial n}{\partial x}~,\label{eq:particle_flux} \\
	q \,&=\, -\chi n \frac{\partial T}{\partial x} \,+\, \frac{\Gamma T}{\gamma - 1} \label{eq:heat_flux}
\end{align}
The adiabatic index $\gamma$ will be set to $5/3$, as is customary for monatomic gases.
It is important to note that these fluxes omit explicit drift velocities as well as particle and heat sources from within the domain of the model \cite{zohm_dynamic_1994}.

The particle and heat diffusivities $D$ and $\chi$ are considered to be functions of the turbulence.
The L-H transition is not expected to be caused by a difference in form between the two diffusivities.
It is therefore assumed they are proportional.
\begin{align} % chi-D relation
	\chi \,=\, \frac{D}{\zeta (\gamma - 1)} \label{eq:heat_particle_diff_relation}
\end{align}
The parameter $\zeta$ determines the coupling strength of the two diffusivities. The specifics of the forms chosen for the diffusivities is discussed in Section~\ref{sec:diffusivities}.

In order to simplify the model, substituting the definition of the internal energy can be made:
\begin{align} % U definition
	U \,\equiv\, \frac{n T}{\gamma - 1} \label{eq:U_definition}
\end{align}
Making this substitution, along with the fluxes, and subsequently working out the product rules of the derivatives, much more compact forms of Equations~\ref{eq:n_continuity} and \ref{eq:U_continuity}.
One convenience is that Equation~\ref{eq:n_continuity} shows up as a collection of terms within Equation \ref{eq:U_continuity}.
This allows those terms to vanish.
The simplified forms of these transport equations is as such:
\begin{align} % Compact transport equations
	\frac{\partial n}{\partial t} \,&=\, \frac{\partial}{\partial x} \left[D \, \frac{\partial n}{\partial x}\right]~,\label{eq:n_compact} \\
	\frac{\partial T}{\partial t} \,&=\, \frac{\partial}{\partial x} \left[\frac{D}{\zeta} \, \frac{\partial T}{\partial x}\right] \,+\, \left(1 + \frac{1}{\zeta}\right) \frac{\partial n}{\partial x} \, \frac{\partial T}{\partial x} \label{eq:T_compact}
\end{align}

\section{Radial Electric Field}\label{sec:Z_equation}
A rigorous theory on what causes the diffusion coefficients $D$ and $\chi$ is still unknown.
However, experiments and mathematical investigations indicate that flows in the plasma can tear apart turbulent eddies, which reduces the transport radially.\todo{\color{red}{Reword?}}
To model the dynamics of the transition fully within this scope, the radial electric field must also be evolved explicitly.

In this investigation, the radial electric field $E_r$ is normalized with the ion poloidal gyroradius $\rho_p$ into $Z$
\begin{align} % Z definition
	Z \,\equiv\, \frac{\rho_p e E_r}{T}~, ~~~~ \rho_p \,\equiv\, \frac{m_i v_\text{th}}{e B_\theta}~,\label{eq:Z_and_rho_definitions}
\end{align}
with the ion mass $m_i$, the thermal ion velocity $v_\text{th}$, the elementary charge $e$, and the poloidal magnetic field $B_\theta$.

Itoh \emph{et al.}, originally modeled an evolution to the radial electric field, which included hysteresis and an oscillatory phase \cite{itoh_edge_1991}
\begin{align} % Original Itoh model
	\frac{\partial Z}{\partial t} \,&=\, -N(Z,g) + \mu \frac{\partial Z^2}{\partial x^2}~,\label{eq:original_z} \\
	N(Z,g) \,&\equiv\, g - g_0 + \left[\beta Z^3 - \alpha Z\right]~.
\end{align}
The nonlinear $N(Z,g)$ function introduces the bifurcating behavior, with $g$ acting as a gradient parameter, and $g_0$, $\alpha$, and $\beta$ as the bifurcation parameters \cite{itoh_model_1988}.
Paquay rewrote the equation as
\begin{align} % Modern model
	\epsilon \, \frac{\partial Z}{\partial t} \,=\, \mu \, \frac{\partial^2 Z}{\partial x^2} \,+\, \frac{c_n T}{n^2} \cdot \frac{\partial n}{\partial x} \,+\, \frac{c_T}{n} \cdot \frac{\partial T}{\partial x} \,+\, G(Z)~,\label{eq:paquay_Z} \\
	G(Z) \,\equiv\, a + bZ + cZ^3 \label{eq:G_polynomial}
\end{align}

\section{Forms of Diffusivities}\label{sec:diffusivities}
Various forms of $D$ (and subsequently $\chi$) are presented in investigations with this model.
The original Itoh model has $D$ proportional to the hyperbolic tangent of the normalized radial electric field \cite{itoh_edge_1991} \cite{zohm_dynamic_1994}
\begin{align} % Itoh diffusivity
	D(Z) \,&=\, \frac{D_\text{max} + D_\text{min}}{2} + \frac{(D_\text{max} - D_\text{min})\tanh(Z)}{2}~. \label{eq:Itoh_diffusivity}
\end{align}

However, it is now accepted that the diffusivity must be a function of the shear of the field.
This is either due to the stabilization of linear modes, or through the reduction of turbulence amplitudes, correlation lengths, or change in phases of the fluctuations \cite{connor_review_2000}.
It was found that ``a flattened (steep) radial equilibrium gradient tends to enhance (eliminate) turbulence suppression due to the shear flow'' \cite{zhang_edge_1992}.
\begin{align} % Shear diffusivity
	D(Z^{\prime}) \,=\, \frac{D_\text{min}}{1 + c(Z^{\prime})^{\gamma}} \label{eq:shear_diffusivity}
\end{align}
Staps uses this form, with $c = 0.5$ and $\gamma = 2$ \cite{staps_backstepping_2017}.
Another form includes both the field itself and its shear \cite{paquay_studying_2012}
\begin{align} % Flow-Shear equation
	D(Z, Z^{\prime}) \,=\, D_\text{min} \,+\, \frac{D_\text{max} - D_\text{min}}{1 + a_1 Z^2 + a_2 Z \cdot Z^{\prime} + a_3 (Z^{\prime})^2} \label{eq:flow-shear}
\end{align}

\section{Boundary Conditions}\label{sec:boundary_conditions}
The set of aforementioned PDE's is evaluated in a domain that is small enough to exclude the inner (core) boundary from any possible particle and heat sources.
In this model, this boundary to the domain is $x = L$, for some positive length $L$, with the core is assumed to be at $x \gg L$.
It also requires the domain to be sufficiently large to fully-encompass the transport barrier and ignore any possible boundary effects of the barrier itself.
The outer boundary of the plasma edge is assumed to be the SOL (scrape-off layer).
This corresponds to $x = 0$.

It is assumed that the SOL is passive to particle and heat fluxes, and therefore the density and temperature profiles exponentially decay.
\begin{align} % n and T boundary conditions at SOL
	\frac{\partial n}{\partial x}\biggr \lvert_{x=0} \,=\, \frac{n}{\lambda_n}~, ~~~~ \frac{\partial T}{\partial x}\biggr \lvert_{x=0} \,=\, \frac{T}{\lambda_T}~.\label{eq:nT_SOL_boundary}
\end{align}
The core side has a constant influx of particles and heat, stemming from fueling and heating.
The electric field is to have no gradient beyond the plasma edge.
\begin{align} % Fluxes from core
	\Gamma(L) \,=\, \Gamma_c~, ~~~~ q(L) \,=\, q_c~,\label{eq:core_fluxes} \\
	\frac{\partial Z}{\partial x}\biggr \lvert_{x=L} \,=\, 0 \label{eq:core_Z_cond}
\end{align}

