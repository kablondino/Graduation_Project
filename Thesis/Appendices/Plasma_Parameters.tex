\chapter{Plasma Parameters}\label{chapter:Plasma_Parameters}

All temperatures are in units of eV, unless otherwise noted, and the subscript $j$ indicates either electrons or ions.

The thermal velocity is chosen as the one that selects the most probable speed.
Without affecting the outcome of the study, one could choose another scheme, as they only differ by a relatively small factor.
\begin{align}
	v_{T_j} \,=\, \sqrt{\frac{2 \, e \, T}{m_j}}
		\label{eq:thermal_velocity}
\end{align}

Safety factor:
\begin{align}
	q \,=\, \frac{a_m \, B_\phi}{R \, B_\theta} \,=\,
		\frac{\epsilon \, B_\phi}{B_\theta} \label{eq:safety_factor}
\end{align}
Note that the symbol used for the aspect ratio is $\epsilon$ without subscript.
Any value of permittivity is indicated by an $\epsilon$ \emph{with} a subscript.

The frequency by which ions go from being in trapped orbits to untrapped orbits is the ion transition angular frequency.
\begin{align}
	\omega_t \,=\, \frac{v_{T_i}}{q \, R} \label{eq:transition_freq}
\end{align}

The frequency by which ions/electrons ``bounce'' from one end of it's banana orbit to the other and back is the banana orbit bounce frequency.
\begin{align}
	\omega_{bj} \,=\, \frac{\epsilon^{3/2} \, v_{T_j}}{q \, R}
		\label{eq:banana_bounce_freq}
\end{align}

The width of the ion banana orbit plays a crucial role in the ion orbit loss flux.
\begin{align}
	w_{bi} \,=\, \rho_{\theta i} \sqrt{\epsilon} \label{eq:banana_width}
\end{align}

Electron-ion and ion-ion collision frequencies:
\begin{align}
	\nu_{ei} \,&=\, 1.33\times 10^5 \, \frac{n_{20}}{(T_\text{keV})^{3/2}}
		\,=\, 4.2058\times 10^{-5} \, \frac{n}{(T_\text{eV})^{3/2}}
		\label{eq:nu_ei} \\
	\nu_{ii} \,&=\, 1.2 \, \nu_{ei} \sqrt{\frac{m_e}{m_i}} \label{eq:nu_ii}
\end{align}

The ion collisionality is defined as the ratio of the ion-ion collision frequency to the banana orbit frequency.
\begin{align}
	\nu_{*i} \,=\, \frac{\nu_{ii}}{\omega_{bi}} \label{eq:collisionality}
\end{align}

The ASDEX-U machine parameters are presented from the source file.
\inputminted[firstline=24, lastline=33, tabsize=4, breaklines=true, fontsize=\footnotesize, frame=single, linenos=true]{python}{../FiPy_Model/parameters.py}

