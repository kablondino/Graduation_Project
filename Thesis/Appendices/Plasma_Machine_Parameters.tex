\chapter{Plasma and Machine Parameters}\label{chapter:Plasma_Parameters}
To tie this investigation to a physical machine, ASDEX-Upgrade was chosen.
Future investigations are aided by this since a wealth of L--H transition data exists for the machine.
All temperatures are in units of eV, unless otherwise noted, and the subscript $j$ indicates either electrons or ions.

The size of the machine is as following, with $a_v = 0.8$ and $a_h = 0.5$ as the vertical and horizontal minor radii, respectively.
\begin{sagesilent}
	reset()
	load("../Sage_Model/parameters.sage")
\end{sagesilent}
\begin{align} % Physical size
	a_m \,&=\, \sqrt{\frac{a_v^2 + a_h^2}{2}} \,\approx\,
		\sage{a_m.n(digits=3)}~\text{m}~,~~~~
		R \,=\, \sage{R.n(digits=3)}~\text{m} \\
		\epsilon \,&=\, \frac{a_m}{R} \,\approx\, \sage{epsilon.n(digits=2)}
\end{align}
Note that the symbol used for the aspect ratio is $\epsilon$ without subscript.
Any value of permittivity is indicated by an $\epsilon$ \emph{with} a subscript.

The magnetic field is produced by both the toroidal coils and the central solenoid.
\begin{align} % Magnetic fields
	I_\phi \,&=\, \sage{(I_phi/1.0e6).n(digits=3)}~\text{MA}~,~~~~
		B_\theta \,=\, \frac{\mu_0 \, I_\phi}{2\pi \, a_m} \,\approx\,
		\sage{B_theta.n(digits=3)}~\text{T}, \\
	B \,&=\, \sage{B.n(digits=4)}~\text{T}
\end{align}
These allow for the calculation of the safety factor.
\begin{align} % Safety factor
	q \,=\, \frac{a_m \, B_\phi}{R \, B_\theta} \,=\,
		\frac{\epsilon \, B_\phi}{B_\theta} \,\approx\, \sage{q.n(digits=3)}
		\label{eq:safety_factor}
\end{align}

The thermal velocity is chosen as the one that selects the most probable speed.
Without affecting the outcome of the study, one could choose another scheme, as they only differ by a relatively small factor.
\begin{align} % Thermal velocity
	v_{T_j} \,=\, \sqrt{\frac{2 \, e \, T}{m_j}}
		\label{eq:thermal_velocity}
\end{align}
The frequency by which ions go from being in trapped orbits to untrapped orbits is the ion transition angular frequency.
\begin{align} % Transition frequency
	\omega_t \,=\, \frac{v_{T_i}}{q \, R} \label{eq:transition_freq}
\end{align}
The frequency by which ions/electrons ``bounce'' from one end of it's banana orbit to the other and back is the banana orbit bounce frequency.
\begin{align} % Banana frequency
	\omega_{bj} \,=\, \frac{\epsilon^{3/2} \, v_{T_j}}{q \, R}
		\label{eq:banana_bounce_freq}
\end{align}
The width of the ion banana orbit plays a crucial role in the ion orbit loss flux.
It heavily affects the penetration depth of this flux.
\begin{align} % Banana width
	w_{bi} \,=\, \rho_{\theta i} \sqrt{\epsilon} \label{eq:banana_width}
\end{align}
Electron-ion and ion-ion collision frequencies are taken from Freidberg \cite{freidberg_plasma_2007}.
\begin{align} % Electron-ion and ion-ion collision frequencies
	\nu_{ei} \,=\, 1.33\times 10^5 \, \frac{n_{20}}{(T_\text{keV})^{3/2}}
		\,&=\, 4.2058\times 10^{-5} \, \frac{n}{(T_\text{eV})^{3/2}}
		\label{eq:nu_ei} \\
	\nu_{ii} \,=\, 1.2 \, \nu_{ei} \sqrt{\frac{m_e}{m_i}} \,&=\,
		\sage{(4.2058e-5* 1.2 * sqrt(constants.m_e / constants.m_p)).n(digits=4)}
		\, \frac{n}{(T_\text{eV})^{3/2}} \label{eq:nu_ii}
\end{align}
The ion collisionality is defined as the ratio of the ion-ion collision frequency to the banana orbit frequency.
\begin{align} % Ion collisionality
	\nu_{*i} \,=\, \frac{\nu_{ii}}{\omega_{bi}} \label{eq:collisionality}
\end{align}

The shape of the neutral density profile $n_0$ is set by the parameters given in Eq.~\ref{eq:neutral_density}.
\begin{align}
	\theta \,=\, 0.1~,~~~ k \,=\, 1000~\text{m}^{-1}~,~~~ d \,=\, 2~\text{cm}
\end{align}
Any proportionality constant that gives the equation proper units of density is set to unity.
%\inputminted[firstline=24, lastline=33, tabsize=4, breaklines=true, fontsize=\footnotesize, frame=single, linenos=true]{python}{../FiPy_Model/parameters.py}

