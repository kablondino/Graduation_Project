\chapter{Code}\label{chapter:Code}
As covered Chapter~\ref{chapter:numerical_method}, the finite volume method solver FiPy \cite{guyer_fipy:_2009} is used to solve the system of equations.
This section will obviously not be comprehensive, but it should give an idea on the manner by which the equations were declared.
\inputminted[firstline=21, lastline=39, tabsize=4, breaklines=true, fontsize=\footnotesize, frame=single, linenos=true]{python}{../FiPy_Model/solving_flux.py}

Because the system is highly nonlinear, it is required to ``sweep'' for the solution.
This is where the solver attempts a solution, and returns a residual.
The user is required to choose a tolerance which the residual must be less than for the solver to move to the next time iteration.
\inputminted[firstline=92, lastline=107, tabsize=4, breaklines=true, fontsize=\footnotesize, frame=single, linenos=true]{python}{../FiPy_Model/solving_flux.py}

