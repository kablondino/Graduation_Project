\chapter{Code}\label{chapter:Code}
As covered Chapter~\ref{chapter:numerical_method}, the finite volume method solver FiPy \cite{guyer_fipy:_2009} is used to solve the system of equations.
This section will obviously not cover the code comprehensively, but it should give an idea on how the equations were declared.

In the main solving program, the declarations of the equations and their coupling is the following:
\inputminted[firstline=22, lastline=43, tabsize=4, breaklines=true, fontsize=\footnotesize, frame=single, linenos=true]{python}{../FiPy_Model/solving_flux.py}
One may notice in line \mintinline{python}{33}, the \mintinline{python}{TransientTerm} declares density $n$ as a coefficient.
This corresponds to the term $\dfrac{\partial (n \, Z)}{\partial t}$; however, as noted earlier, the timescale by which the density and temperature evolves is much slower than that of the radial electric field.
The extra term associated with it is therefore much smaller than the one written in on the left-hand side of Eq~\ref{eq:reduced_normalized_Z_equation}.

Because the system is highly nonlinear, it is required to ``sweep'' for the solution.
This is where the solver attempts a solution, and returns a residual.
The user is required to choose a tolerance which the residual must be less than for the solver to move to the next time iteration.
\inputminted[firstline=98, lastline=113, tabsize=4, breaklines=true, fontsize=\footnotesize, frame=single, linenos=true]{python}{../FiPy_Model/solving_flux.py}

The \mintinline{python}{GMRES_Solver} option in the sweep indicates that the Generalized Minimal Residual Method was chosen for solving.
It was most consistent in converging the system.
The residual tolerance for the system in SI units was set to $10^{14}$.

