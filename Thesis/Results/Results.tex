\chapter{Results}\label{chapter:results}
This chapter is broken into two sections.
The first covers verification to any of the changes made to the numerical parameters of the original model of the $Z$ equation (Eq.~\ref{eq:original_Z_equation}).
The second section shows the results of the flux version, Eq.~\ref{eq:reduced_normalized_Z_equation}, including the problems that arose.

\section{Original Model} \label{sec:original_results}
Also as proven in \cite{paquay_studying_2012}, the Flow-Shear model of the diffusivity Eq.~\ref{eq:flow_shear_diffusivity} generates the bifurcating behavior as expected.
Fig.~\ref{fig:Original_FS} shows L--mode as the initial condition, with a clear transport barrier formed, shown in this case at $x = 1.0$.
Not shown here for the sake of brevity, a steady-state L--mode was also obtained through lowering the core fluxes.
%{{{ Two time slices in the Flow-Shear Original Model
\TwoFigOneCap{\includegraphics[width=\textwidth]{../Graphics/Model_Graphs/Original_FS_0000.png}}
	{\includegraphics[width=\textwidth]{../Graphics/Model_Graphs/Original_FS_0135.png}}
	{The left plot shows the initial conditions set forth by Eqs.~\ref{eq:n_initial}--\ref{eq:T_initial}, with the right plot at a later time.
	The diffusivity was set to the Flow-Shear model, \emph{i.e.} Eq.~\ref{eq:flow_shear_diffusivity}, with $a_1 = 0.1$, $a_2 = 0$, and $a_3 = 0.5$.
	The core particle and heat fluxes, respectively, are set to $-0.8$ and $4.0$.}
	{fig:Original_FS}
%}}}
The sign of $Z$ may differ from previous investigations; the choice of polynomial parameters and gradient coefficients are responsible.
However, since the diffusivity is an even function, it does not affect the overall operational mode.
The original diffusivity of Eq.~\ref{eq:Itoh_diffusivity}, an odd function, was not explored here, and is considered out-dated.

If we look at the steady-state values of the $Z$ equation, we can identify stationary points, akin to what is done in Fig.~\ref{fig:stationaries_b}.
In Fig.~\ref{fig:original_stationary_vs_Z}, the steady-state is plotted at three different spots in the domain, corresponding to the simulation shown in Fig.~\ref{fig:Original_FS}.
The three spots are at the edge, the center of the high-shear area, and in the middle of the domain.
These combined areas can be considered to show both L-- and H--mode.
In addition, two more lines are added, indicating the first time step, and the polynomial $G(Z)$.
\begin{figure}[tb] % Steady-State as a function of Z, original model
	\centering
	\begin{sagesilent}
		reset()

		var('Z')

		c_n, c_T = -1.1, -0.9
		a, b, c = 3.0/2.0, 2.0, -1.0
		Z_S = -3.0/2.0

		# At t = 0
		#0.49	0.683706980466396	1.32496145938483	-0.0206250867984561	5
		#0.51	0.691213814264995	1.33027412776876	-0.0191672209048899	5
		#0.53	0.698719562491543	1.33554223315998	-0.0177733476721204	5

		# At t = 135
		#0.49	0.610660563155032	1.30411254952973	-3.06036908428267	2.66118461102601
		#0.51	0.618549890710752	1.31299641246706	-3.05151915868033	2.66386371895
		#0.53	0.626467999984115	1.32173998954979	-3.0425333319374	2.66544015688424
		# .........
		#0.99	0.981110868143376	1.63421628926577	-1.48108678379191	0.491444796219089
		#1.01	0.999673727872353	1.64981311266731	-1.28285663026662	0.499537912103461
		#1.03	1.01660486587383	1.66401200235489	-1.09300528560897	0.513478891305364
		# .........
		#1.99	1.2435198107485	1.79762273772508	0.141694442025275	4.9908892834125
		#2.01	1.24801940647034	1.79934360213636	0.141647858350999	4.99089991282975
		#2.03	1.2525541146463	1.80107077486164	0.141626086988019	4.99090015667972

		t = (0, 135, 135, 135)
		x = (0.51, 0.51, 1.01, 2.01)
		density = (0.691213814264995, 0.618549890710752, 0.999673727872353, 1.24801940647034)
		density_grad = ((0.698719562491543 - 0.683706980466396) / 0.04,\
				(0.610660563155032 - 0.626467999984115) / 0.04,\
				(1.01660486587383 - 0.981110868143376) / 0.04,\
				(1.2525541146463 - 1.2435198107485) / 0.04)
		temperature = (1.33027412776876, 1.31299641246706, 1.64981311266731, 1.79934360213636)
		temperature_grad = ((1.33554223315998 - 1.32496145938483) / 0.04,\
				(1.32173998954979 - 1.30411254952973) / 0.04,\
				(1.66401200235489 - 1.63421628926577) / 0.04,\
				(1.80107077486164 - 1.79762273772508) / 0.04)

		colors = rainbow(len(x) + 1)

		G(Z) = a + b*(Z - Z_S) + c*(Z - Z_S)^3

		steady_state_plots = []

		for j in range(len(x)):
		    f(Z) = (c_n * temperature[j] / density[j]^2) * density_grad[j] + (c_T / density[j]) * temperature_grad[j] + G(Z)
		    steady_state_plots.append(plot(f, (Z,-3.5,1.5), ymin=-3, ymax=4,\
		    		gridlines=True, color=colors[j], axes_labels=["$Z$",""],\
		    		thickness=2.0, legend_label="$x = " +str(x[j].n(digits=3))\
		    		+"$, $t = " +str(t[j].n(digits=3))+ "$"))

		steady_state_plots.append(plot(G, (Z,-3.5,1.5), gridlines=True,\
				color=colors[-1], thickness=2.0, legend_label="$G(Z)$"))

		equation_label = text(r"$f(Z) = \frac{c_n \, T}{n^2} \, \frac{\partial n}{\partial x} \,+\, \frac{c_T}{n} \, \frac{\partial T}{\partial x} \,+\, G(Z)$",\
				(-2,-2.5), bounding_box={'boxstyle': 'round', 'fc': 'w'},\
				fontsize=16, color='black')

		combined_plots = sum(steady_state_plots) + equation_label
		combined_plots.set_aspect_ratio(0.4)
	\end{sagesilent}
	\sageplot[width=0.9\textwidth]{combined_plots}
	\caption{This plot shows the steady-state values across $Z$ at different locations in the domain, corresponding with the plots in Fig.~\ref{fig:Original_FS}.
	It assumes the diffusion term of the field is negligible.
	The three locations correspond to near the edge, at the transport barrier, and half-way in the domain.
	In addition, the first time step, considered in L--mode everywhere, and the polynomial $G(Z)$ are also added for comparison.}
	\label{fig:original_stationary_vs_Z}
\end{figure}

In regions of low gradient values, the function is shifted upwards, causing only one stable point to form.
An increase in gradient values causes a decrease to these functions.
The lowest line is the one at the barrier, showing three zeros, with the middle being unstable.
The left zero of about $-3$ provokes the edge-side $Z$ value in the state.

\section{Flux Model} \label{sec:flux_results}
The flux model was simulated using SI units, with the routine exception of temperature in electron-volts.
Scans over input power ($\Gamma_c$ and $q_c$) were done with both the full model and with excluding one nonambipolar flux at a time.
However, one currently-unexplained phenomenon commanded the behavior.

First, a full run of the model was done, with the relevant variables shown in Fig.~\ref{fig:flux_state_full}.
The parameters of the diffusivity, the time step size, and the core particle flux are indicated in the title.
This time step was chosen to show the implying transport barrier characteristic of H--mode.
Table~\ref{table:flux_values} shows particular components of the fluxes at the same time step, when applicable, with the totals on the rightmost column.
%{{{ Two Figures for showing the full state, pre-oscillations
\TwoFigOneCap{\includegraphics[width=\linewidth]{../Graphics/Model_Graphs/state_G1e20_full_t200.png}}
	{\includegraphics[width=\linewidth]{../Graphics/Model_Graphs/flux_G1e20_full_t200.png}}
	{The state including all of the nonambipolar fluxes.
	The parameters for the diffusivity function, the core flux, and the particular time step are indicated.
	}
	{fig:flux_state_full}
%}}}

\begin{table}[!htb] % Shows flux data for a time slice at two locations
	\centering
	\small\begin{tabular}{c|r|ccccc}
	\begin{sagesilent} # At x ~ 0.002, t = 200
		reset()
		load("../Sage_Model/parameters.sage")
		# THE DATA USED IN THE NEXT TABLE
		#x	$n$	$T$	$Z$	$D$	$D_{an}$	$\Gamma_e^{an}$	$n_0$	$\langle\sigma_{cx} v\rangle$	$\Gamma_i^{cx}$	$D_{\pi\parallel}$	$\Gamma_i^{\pi\parallel}$	$\Gamma_i^{ol}$
		t = 200
		x, density, temperature, Z, Diffusivity, D_an, Gamma_an, n_0, cx_rate, Gamma_cx, D_bulk, Gamma_bulk, Gamma_ol = 0.002125, 5.35434214493223e+17, 165.427742871548, 1.47166341450206, 2.13128525822014, 0.000700479137552669, 2.30504108694339e+17, 56167794581565.9, 2.5481649909684e-13, -4.83290085496657e+16, -0.0191697028228131, -1.17180463596314e+18, -466734643459656

		v_Ti = sqrt(2.0 * constants.e * temperature / constants.m_p)
		rho_pi = constants.m_p * v_Ti / (constants.e * B_theta)

		nu_ii = 1.2 * sqrt(constants.m_e / constants.m_p)\
				* 4.2058e-11 * density / sqrt(temperature^3)
		omega_t = v_Ti / (q * R)

		g_n_cx = -constants.m_p*n_0*cx_rate*density*(temperature/constants.e)\
				/ (B_theta^2) * (B_theta^2 / (aspect*B_phi)**2)
		g_T_cx = g_n_cx * alpha_cx
		g_Z_cx = g_n_cx / rho_pi
	\end{sagesilent}
		$x$ & & $\dfrac{n^\prime}{n}$ & $\dfrac{T^\prime}{T}$ & $Z$ & Total & $J_j^\text{k}$ \\ \hline
		\multirow{4}{*}{$\sage{(x*100).n(digits=3)}$ cm} & $\dagger \, \Gamma_i^{\pi\parallel}$ & $\sage{(density * D_bulk * sqrt(pi)*exp(-Z^2)).n(digits=3)}$ & --- & $\sage{((density * D_bulk / rho_pi) * sqrt(pi)*exp(-Z^2)).n(digits=3)}$ & $\sage{Gamma_bulk.n(digits=3)}$ & $\sage{(constants.e * Gamma_bulk).n(digits=3)}$ \\
		& $\Gamma_e^\text{an}$ & $\sage{(density * D_an).n(digits=3)}$ & $\sage{(density * D_an * alpha_an).n(digits=3)}$ & $\sage{(density * D_an / rho_pi).n(digits=3)}$ & $\sage{Gamma_an.n(digits=3)}$ & $\sage{(constants.e * Gamma_an).n(digits=3)}$ \\
		& $\Gamma_i^\text{cx}$ & $\sage{g_n_cx.n(digits=3)}$ & $\sage{g_T_cx.n(digits=3)}$ & $\sage{g_Z_cx.n(digits=3)}$ & $\sage{Gamma_cx.n(digits=3)}$ & $\sage{(constants.e * Gamma_cx).n(digits=3)}$ \\
		& $\Gamma_i^\text{ol}$ & --- & --- & --- & $\sage{Gamma_ol.n(digits=3)}$ & $\sage{N(constants.e * Gamma_ol).n(digits=3)}$ \\ \hline

	\begin{sagesilent} # x ~ 0.030125, t = 200
		x, density, temperature, Z, Diffusivity, D_an, Gamma_an, n_0, cx_rate, Gamma_cx, D_bulk, Gamma_bulk, Gamma_ol = 0.030125, 1.08615114035334e+18, 221.843028685801, 1.77821178953312, 4.94743894562273, 0.00108811581270749, 6.12848344919829e+17, 1943024953.06425, 2.81018035060206e-13, -4232723648741.13, -0.0310869925862421, -1.37764340545142e+18, -1.15809327369854e-59

		# Update for new values
		v_Ti = sqrt(2.0 * constants.e * temperature / constants.m_p)
		rho_pi = constants.m_p * v_Ti / (constants.e * B_theta)

		nu_ii = 1.2 * sqrt(constants.m_e / constants.m_p)\
				* 4.2058e-11 * density / sqrt(temperature^3)
		omega_t = v_Ti / (q * R)

		g_n_cx = -constants.m_p*n_0*cx_rate*density*(temperature/constants.e)\
				/ (B_theta^2) * (B_theta^2 / (aspect*B_phi)**2)
		g_T_cx = g_n_cx * alpha_cx
		g_Z_cx = g_n_cx / rho_pi
	\end{sagesilent}

		\multirow{4}{*}{$\sage{(x*100).n(digits=3)}$ cm} & $\dagger \, \Gamma_i^{\pi\parallel}$ & $\sage{(density * D_bulk * sqrt(pi)*exp(-Z^2)).n(digits=3)}$ & --- & $\sage{((density * D_bulk / rho_pi) * sqrt(pi)*exp(-Z^2)).n(digits=3)}$ & $\sage{Gamma_bulk.n(digits=3)}$ & $\sage{(constants.e * Gamma_bulk).n(digits=3)}$ \\
		 & $\Gamma_e^\text{an}$ & $\sage{(density * D_an).n(digits=3)}$ & $\sage{(density * D_an * alpha_an).n(digits=3)}$ & $\sage{(density * D_an / rho_pi).n(digits=3)}$ & $\sage{Gamma_an.n(digits=3)}$ & $\sage{(constants.e * Gamma_an).n(digits=3)}$ \\
		 & $\Gamma_i^\text{cx}$ & $\sage{g_n_cx.n(digits=3)}$ & $\sage{g_T_cx.n(digits=3)}$ & $\sage{g_Z_cx.n(digits=3)}$ & $\sage{Gamma_cx.n(digits=3)}$ & $\sage{(constants.e * Gamma_cx).n(digits=3)}$ \\
		 & $\Gamma_i^\text{ol}$ & --- & --- & --- & $\sage{Gamma_ol.n(digits=3)}$ & $\sage{(constants.e * Gamma_ol).n(digits=3)}$
	\end{tabular}\normalsize
	\caption{This table shows the values of the nonambipolar fluxes at two locations in the domain.
	The upper half comes from a region of the transition, while the lower half is close to the core of the domain, indisputably in L--mode.
	The appropriate gradient coefficients are shown in the columns, \emph{i.e.} $g_l^\text{k}$.
	Note that, since bulk viscosity $\Gamma_i^{\pi\parallel}$ has a nonlinear term in the plasma dispersion function, its gradient coefficients are not entirely comparable.
	Nevertheless, they do give an indication of the relative dominance.
	The time step used is the same of that in Fig.~\ref{fig:flux_state_full}.}
	\label{table:flux_values}
\end{table}
One note from this data is that the neoclassical bulk viscosity $\Gamma_i^{\pi\parallel}$ results in the largest flux.
This is found consistently throughout all combinations of fluxes and input power.

\begin{figure}[htb] % FLUXES STEADY-STATE, SAVED IMAGE, because it takes to long to run
	\centering
	\includegraphics[width=0.9\textwidth]{../Graphics/Model_Graphs/Flux_vs_Z_t200_core_simple.png}
	%\includegraphics[width=0.9\textwidth]{../Graphics/Model_Graphs/Flux_vs_Z_t200_core_complex.png}
	\caption{This plots is akin to Fig.~\ref{fig:original_stationary_vs_Z} for the flux model; each nonambipolar flux is plotted against $Z$ at the indicated location and time.
	It clearly shows three steady-state points, just as the original model, and shows the dominance of the bulk viscosity.}
\end{figure}
For absolute values above some threshold, the bulk viscosity defined with the full plasma dispersion function becomes erratic and unphysical.
It seems to be similar to the function $\sin(1/x)$ as $x$ approaches zero.

%{{{ Two Figures for displaying the oscillations with large time step
\TwoFigOneCap{\includegraphics[width=\textwidth]{../Graphics/Model_Graphs/state_G1e20_full_t400.png}}
	{\includegraphics[width=\textwidth]{../Graphics/Model_Graphs/state_G1e20_full_t401.png}}
	{These two plots are one time step from another, with the same parameters as those found in Fig.~\ref{fig:flux_state_full}.
	The defining feature of the mathematics is the rapid oscillations that occur.}
	{fig:show_oscillations}
%}}}
%{{{ Two Figures for displaying the oscillations with short time step
\TwoFigOneCap{\includegraphics[width=\textwidth]{../Graphics/Model_Graphs/state_G1e20_extended_t1059.png}}
	{\includegraphics[width=\textwidth]{../Graphics/Model_Graphs/state_G1e20_extended_t1060.png}}
	{The same simulation was done as in Fig.~\ref{fig:show_oscillations}, with a time step one-tenth the size.}
	{fig:show_oscillations_extended}
%}}}

On the note of bifurcation: due to the linear forms of the anomalous electron loss $\Gamma_e^\text{an}$ and charge exchange friction $\Gamma_i^\text{cx}$, bifurcating behavior cannot arise due to them.

