\chapter{Conclusion}\label{chapter:conclusion}

\begin{itemize}
	\item \textbf{Which electric field-generating mechanisms are dominant in concrete experimental tokamak conditions?}

	Bulk viscosity consistently has the largest effect on the field.
	Ion orbit loss becomes highly suppressed once there is a nonzero value of the electric field at the edge.
\end{itemize}


\begin{itemize}
	\item \textbf{What mechanisms, and therefore necessary parameters, should be investigated?}
\end{itemize}


\begin{itemize}
	\item \textbf{What are the appropriate forms of each flux for the model and what are the relative sensitivities and sources of uncertainty?}

	Ion orbit loss $\Gamma_i^\text{ol}$ is highly suppressed by nonzero values of $Z$.
	Also, due to its high spatial sensitivity, this also pins it to the edge.
	Uncertainties in the position or value of $Z$ highly affect this term, but does not manifest in any significant change in $Z$ itself.

	The shear viscosity $J^{\pi\perp}$ is sensitive only to the dynamic viscosity $\mu$, and is completely necessary to have the system at all.

	The parameters of the diffusivity are extremely sensitive, as a small change to the coefficient of the shear viscosity $a_3$ can change the fundamental behavior of the entire system.
	Due to this being a qualitative model, in which the parameters are chosen rather than calculated,...
\end{itemize}


\begin{itemize}
	\item \textbf{What is the variation in dominance of each field-generating mechanism across increasing input power?}

	As of now, charge exchange friction is directly proportional to the core particle flux $\Gamma_c$, giving it artificially-high relation on initialization.
	However, the bulk viscosity $\Gamma_i^{\pi\parallel}$
\end{itemize}

