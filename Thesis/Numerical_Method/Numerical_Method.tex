\chapter{Numerical Method}\label{sec:numerical_method}
\todo{\color{red}Whole chapter? Or just section?}
The finite volume method (FVM) is a numerical method of evaluating and representing partial differential equations.
It is sometimes, confusingly, referred to as the finite difference \emph{scheme} or the cell-centered difference scheme, the former which clashes in name with the finite difference method.
There are some characteristics it has in common with the two other well-known schemes, the finite difference method and the finite element method (FEM).

All three methods involve nodes generated on a discrete mesh.
In the FVM, the nodes are surrounded by the discrete elements called control volumes, with space variables able to be defined at the volume's centers or faces.
In solving, volume integrals that contain a divergence term are converted by the divergence theorem to surface integrals.
The terms are evaluated as fluxes at the surfaces of each control volume, and are accounted for between volumes, \emph{i.e.} conserved.
It also allows for discontinuities of the coefficients, assuming it occurs on the boundaries of the volumes.
Equations \ref{eq:shear_diffusivity} and \ref{eq:flow_shear_diffusivity} contain electric field shear term(s) to some exponent, potentially generating these discontinuities.
This makes the method popular for computational fluid dynamics, of which one could consider this investigation \cite{eymard_finite_2003}.

%%%%%%%%
%The FEM is based on the variational formulation of the problem; it is obtained by multiplying the original equation by a test function.
%The unknown variable is then approximated by some linear combination of ``shape'' functions, which then the equation in each element is integrated \cite{eymard_finite_2003}.
%%%%%%%%

