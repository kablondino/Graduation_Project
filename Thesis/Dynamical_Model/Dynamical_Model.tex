\chapter{Dynamical Model}\label{chapter:dynamical_model}
Fusion plasmas are inherently complicated systems, and thus are complicated to model, even when taking simplification liberties.
The model developed by Itoh \emph{et al.} \cite{itoh_edge_1991} and Zohm \cite{zohm_dynamic_1994}, and later refined by others, assumes that the transport barrier occurs in a thin layer at the plasma edge.
The plasma edge is defined to be at the last-closed flux surface (separatrix).
This allows for slab geometry to be used, in which $\psi$ is a usual radial coordinate.
However, $x$ will be used as it is more distinctly recognized as a spatial coordinate.
The slab geometry reduces the model into a system of one (spatial) dimensional partial differential equations.
The domain of the model is $0 \,\leq\, x \,\leq\, L$, in which $x = 0$ is the plasma outer edge, and $x = L$ is some depth towards the core.
This differs from past investigations, with $-L \,\leq\, x \,\leq\, 0$, with $x = 0$ also as the outer edge.

\section{Transport Equations}\label{sec:transport_eqs}
To model the transport of a fully-ionized fusion plasma, conservation of mass and energy are considered.
These lead to continuity equations of plasma density $n$ and internal energy $U$
\begin{align} % n and U continuity
	\frac{\partial n}{\partial t} \,+\, \frac{\partial \Gamma}{\partial x} \,&=\, 0~,\label{eq:n_continuity} \\
	\frac{\partial U}{\partial t} \,+\, \frac{\partial q}{\partial x} \,&=\, 0\label{eq:U_continuity}~.
\end{align}
The particle flux $\Gamma$ is determined by some effective particle diffusion $D$ due to anomalous transport.
Note that $\Gamma$ here refers to all species of particle fluxes, both ambipolar and nonambipolar.
The heat flux $q$, however, is determined by effective heat convection.
This is the sum of heat diffusion $\chi n T^\prime$, and advection $\Gamma T$.
\begin{align} % Full Particle and Heat Fluxes
	\Gamma \,&=\, -D \, \frac{\partial n}{\partial x}~,
		\label{eq:particle_flux} \\
	q \,&=\, -\chi \, n \, \frac{\partial T}{\partial x} \,+\,
		\frac{\Gamma \, T}{\gamma - 1} \label{eq:heat_flux}
\end{align}
The adiabatic index $\gamma$ will be set to $5/3$, as is customary for monatomic gases / plasmas.
It is important to note that these fluxes omit explicit drift velocities as well as particle and heat sources from within the domain of the model \cite{zohm_dynamic_1994}.

The particle and heat diffusivities $D$ and $\chi$ are considered to be functions of the turbulence.
The L-H transition is not expected to be caused by a difference in form between the two diffusivities.
It is therefore assumed they are proportional as 
\begin{align} % chi-D relation
	\chi \,=\, \frac{D}{\zeta (\gamma - 1)} \label{eq:heat_particle_diff_relation}~.
\end{align}
The parameter $\zeta$ determines the coupling strength of the two diffusivities.
It is customarily set to 0.5, giving the relation of $\chi = 3 D$.
The specifics of the forms chosen for the diffusivities is discussed in Section~\ref{sec:diffusivity}.

In order to simplify the model, substituting the following definition of the internal energy density should be made.
\begin{align} % U definition
	U \,\equiv\, \frac{n \, T}{\gamma - 1} \label{eq:U_definition}
\end{align}
In addition, one of the liberties taken to simplify the model is that a single temperature is used, \emph{i.e.} the electron and ion temperatures are assumed identical.
This is certainly not the most realistic of choices, as ECRH obviously heats electrons more than ions.
Nevertheless, this is not always the case, as it becomes less pronounced with higher densities \cite{sauter_l--_2012}.

Making this substitution, along with the particle and heat fluxes, we can arrive at abridged forms of Equations~\ref{eq:n_continuity} and \ref{eq:U_continuity}.
\begin{align} % Compact transport equations
	\frac{\partial n}{\partial t} \,&=\, \frac{\partial}{\partial x} \left[D \,
		\frac{\partial n}{\partial x}\right]~,\label{eq:n_compact} \\
	\frac{\partial(n\,T)}{\partial t} \,&=\, \frac{\partial}{\partial x}
		\left[\frac{D\,n}{\zeta} \, \frac{\partial T}{\partial x}\right]
		\,+\, \frac{\partial}{\partial x} \left[D\,T \,
		\frac{\partial n}{\partial x}\right]~. \label{eq:U_compact}
\end{align}

The product rules of the derivatives can subsequently be worked out to obtain a further-reduced form.
One convenience is that Equation~\ref{eq:n_continuity} shows up as a collection of terms within Equation \ref{eq:U_compact}.
This allows those terms to vanish:
\begin{align} % More reduced temperature equation
	\frac{\partial T}{\partial t} \,&=\, \frac{\partial}{\partial x}
		\left[\frac{D}{\zeta} \, \frac{\partial T}{\partial x}\right] \,+\,
		\left(1 + \frac{1}{\zeta}\right) \frac{\partial n}{\partial x} \,
		\frac{\partial T}{\partial x}~. \label{eq:T_compact}
\end{align}
Due to the clearly-defined diffusion terms, Eq.~\ref{eq:U_compact} is the form implemented in the solver.

\section{Nonambipolar Particle Fluxes}\label{sec:nonambipolar_fluxes}
As described in Section~\ref{ssec:E_r}, the dynamics of the radial electric field will be determined by investigating the various nonambipolar fluxes.
Particular turbulent and neoclassical phenomena affect ions and electrons differently.
Ultimately, these effects lead to some form of charge separation, inducing a nonambipolar particle flux.
The fluxes chosen for this analysis comes from a compilation by Callen \cite{callen_toroidal_2009}, Itoh and Itoh \cite{itoh_role_1996}, and Stringer \cite{stringer_explanation_1993}.

For added context, the profiles of these fluxes will also be given with abstract L-- and H--mode density, temperature, and $Z$ profiles, given in Figures~.\todo{\color{red}FIGURES MAYBE??!}

\subsection{Polarization Current}\label{ssec:polarization_current}
In Maxwell's correction of Amp\`ere's law, a current is induced whenever the time derivative of an electric field is non-zero in a medium, which originates from the changing state of polarization.
The bounced-average motion of trapped ions has been shown to have a radial drift, on top of the toroidal precession.
Macroscopically, a radial current must flow to provide the torque that changes the angular momentum, and is referred to as the neoclassical polarization current \cite{hinton_neoclassical_1984}.
\begin{align} % Polarization current
	J^\text{pol} \,&=\, \frac{\rho}{B_\theta^2} \,
		\frac{\partial E_r}{\partial t} \,=\, \sum_j \frac{m_j \, n_j }
		{B_\theta^2} \, \frac{\partial E_r}{\partial t} \\
	\,&\approx\, \frac{m_i \, n}{B_\theta^2} \, \frac{\partial E_r}{\partial t}
		\label{eq:polarization_current_original}
\end{align}
This is larger than the classical polarization current by a factor of $B^2 / B_\theta^2 \sim 10^1$.
The total mass density $\rho$ is summed over all the plasma species.
Because electrons are $\approx 2000$ times less massive than ions, they are ignored.
The referenced definition differs by a factor of $c^2$.

It is written with the normalized electric field as the following, with the steps in between shown in Appendix \ref{chapter:Normalization}.
\begin{align} % Normalized polarization current
	J^\text{pol} \,&=\, \frac{e \, n \, \rho_{\theta i}}{2} \,
		\frac{\partial Z}{\partial t} \label{eq:polarization_current_normalized}
\end{align}


\subsection{Shear Viscosity}\label{ssec:shear_viscosity}
The ion shear (perpendicular) viscosity is the perpendicular component of the viscous stress tensor.
In the L--H transition, it can be viewed as coupling the L-- and H--mode solutions spatially.
It is expressed by Itoh \emph{et al.} \cite{itoh_elmy_1993} as
\begin{align} % Shear Viscosity Current
	J^{\pi\perp} \,=\, \nabla \cdot \left[\frac{e \, \mu \, n \,
		\rho_{\theta i}}{v_{T_i}} \, \nabla\left(\frac{E_r}{B_\theta}\right)
		\right]
		\,=\, \nabla \cdot \left[\frac{m_i \, \mu \, n}{B_\theta^2} \,
		\nabla E_r\right]~. \label{eq:shear_visc_current_definition}
\end{align}
Reducing this expression to the one radial direction, and introducing the perpendicular dielectric permittivity $\epsilon_\perp$ \cite{kiviniemi_numerical_2001} gives
\begin{align} % Current with perpendicular permittivity
	J^{\pi\perp} \,=\, e\,\Gamma_i^{\pi\perp} \,&=\,
		-\epsilon_0\,\epsilon_\perp \frac{\partial}{\partial x}
		\left(\mu_i \, \frac{\partial E_r}{\partial x}\right)~, \\
	\epsilon_\perp \,&=\, 1 + \sum_j \frac{m_j \, n_j}{\epsilon_0 \, B_\theta^2}~.
		\label{eq:perp_permittivity}
\end{align}
If one assumes that $n_i \approx n_e$ and $m_i \gg m_e$, then the sum for Eq.~\ref{eq:perp_permittivity} is only the ion term and the shear viscosity current is written as
\begin{align} % Definition of perpendicular permittivity
	J^{\pi\perp} \,=\, -\frac{m_i}{B_\theta^2} \,
		\frac{\partial}{\partial x} \left[\mu \, n \,
		\frac{\partial E_r}{\partial x}\right]~. \label{eq:shear_current_E_r}
\end{align}
Another liberty taken is assuming the ion kinematic viscosity $\mu$ is a constant in space and time.
However, it will still be written here as though it is not constant.
Normalizing this expression such that it is expressed in terms of $Z$ results in
\begin{align} % Final result of shear viscosity current, normalized
	J^{\pi\perp} \,&=\, -\frac{e \, \rho_{\theta i}}{2} \,
		\frac{\partial}{\partial x} \left[\mu \, n \, \frac{\partial Z}
		{\partial x}\right] \label{eq:shear_current_normalized}
\end{align}
Note that the temperature gradient and concavity has traditionally been ignored for this flux.
Refer to Appendix \ref{chapter:Normalization} to see the derivation of this normalized form.


\subsection{Bulk Viscosity}\label{ssec:bulk_viscosity}
When a compressible fluid experiences expansion without shear, it can still exhibit some internal friction.
This friction is referred to as the bulk or volume viscosity.
The neoclassical bulk viscosity current originates in the different trajectories between electrons and ions due to their mass difference and inhomogeneity in the magnetic field along the field lines \cite{kobayashi_model_2017}.
Two mathematical forms of the parallel ion (bulk) viscosity have been developed in the literature.
Shaing \emph{et al.} \cite{shaing_bifurcation_1990} derives an expression directly for the viscosity by solving the drift kinetic equation with mass flow velocity.
\begin{align}% Shaing Bulk Viscosity
	\langle \mathbf{B}_\theta \cdot \nabla \cdot \boldsymbol{\pi} \rangle \,=\,
		\frac{\epsilon^2 \, m_i \, B \sqrt{\pi}}{4} \, \frac{n\,v_{T_i}}{x}
		\left(I_\theta U_\theta \,+\, I_\phi U_{\theta 0}\right)
		\label{eq:shaing_bulk}
\end{align}
Here, $\boldsymbol{\pi}$ is the ion viscosity tensor, and $U_\theta$ and $U_{\theta 0}$ indicate poloidal flow velocities.
The terms $I_\phi$ and $I_\theta$ represent large poloidal and toroidal integrals, respectively, which are quite taxing to compute.

However, in an effort to circumvent the direct calculations of flow velocities, a different form is introduced by Stringer \cite{stringer_explanation_1993}.
\begin{align} % Stringer Bulk Viscosity
	\Gamma_i^{\pi\parallel} \,=\, \,n_e\,&D_{\pi\parallel}
		\left(\frac{n^\prime}{n} + \frac{Z}{\rho_{\theta i}}\right) \,
		\text{Im}\left[X\left(Z \,+\, i\,\frac{\nu_{ii}}{\omega_t}\right)\right]
		\label{eq:stringer_Gamma_bulk} \\
	&D_{\pi\parallel} \,=\, \frac{\epsilon^2\,\rho_{\theta i}\,T}
		{(x - a_m)\sqrt{\pi}\,B} \label{eq:stringer_D_bulk}
\end{align}
The particle diffusivity for this process is $D_{\pi\parallel}$, which is on the order of $10^{-2}$~m$^2 / $s.
The ion-ion collision frequency is $\nu_{ii}$, the ion angular transition frequency is $\omega_t$, and the mean minor radius is $a_m$.
These parameters are defined in Appendix \ref{chapter:Plasma_Parameters}.

The complex-valued function $X(z)$ is the plasma dispersion function.
It appears in the dispersion equation for linearized waves in a non-relativistic plasma when the velocity distribution is Maxwellian \cite{fried_plasma_2015}.
\begin{align} % Plasma Dispersion Function
	X(z) \,\equiv\, i\,\sqrt{\pi} \, e^{-z^2} \, \text{erfc}(-i\,z) \,=\,
		\frac{1}{\sqrt{\pi}} \int_{-\infty}^{+\infty} \frac{e^{-t^2}}{t - z}
		\, \text{d}t \label{eq:plasma_disp}
\end{align}
\emph{\textbf{Note}} that this function is usually denoted as $Z$, but I am breaking this standard notation to avoid ambiguity with the normalized radial electric field.

In simulating the dynamics, oscillations in the radial electric field began occurring with with frequency much larger than any chosen time step size.
This adjustment was handled by eliminating any complex-valued function in the definition of this nonambipolar flux.
\begin{figure}[tb] % Comparison of complex-valued plasma dispersion with simple gaussian
	\centering
	\begin{sagesilent}
	reset()

	var('x,z')

	plasma_disp(z) = I*sqrt(pi)*exp(-z^2) * erfc(-I*z)

	complex_one(x) = imag(plasma_disp.subs(z=x + 1*i))
	complex_onetenth(x) = imag(plasma_disp.subs(z=x + 0.2*i))
	complex_hundredth(x) = imag(plasma_disp.subs(z=x + 0.05*i))

	complex_one_plot = plot(complex_one, (x,-4,4), color='olive', thickness=2, legend_label=r"Im$[X(Z + 1i)]$")
	complex_onetenth_plot = plot(complex_onetenth, (x,-4,4), thickness=2, color='green', legend_label=r"Im$[X(Z + 0.2i)]$")
	complex_hundredth_plot = plot(complex_hundredth, (x,-4,4), thickness=2, color='turquoise', legend_label=r"Im$[X(Z + 0.05i)]$")

	non_complex_plot = plot(sqrt(pi)*exp(-(x)^2), (x,-4,4), color='blue', thickness=2, legend_label=r"$\sqrt{\pi} \, \exp(-Z^2)$", axes_labels=['$Z$', ''], gridlines=True)

	combined_plots = complex_one_plot + complex_onetenth_plot + complex_hundredth_plot + non_complex_plot
	combined_plots.set_aspect_ratio(2)
#	combined_plots.set_legend_options(font_size=40)
#	combined_plots.show(figsize=[20,10], fontsize=32)
	\end{sagesilent}

%	\includegraphics[width=0.9\textwidth]{../Graphics/Model_Graphs/Bulk_Definition_Comparison.png}
	\sageplot[width=\textwidth]{combined_plots}
	\caption{A visualization of the plasma dispersion function under the condition of Eq.~\ref{eq:plasma_disp_reduction}.
	As the imaginary portion of the argument tends towards zero, the complementary error function with the imaginary argument tends towards unity.}
	\label{fig:bulk_definition_comparison}
\end{figure}

In the limit as $\nu_{ii} / \omega_t$ goes to zero, the plasma dispersion function reduces to a Gaussian, shown in Figure~\ref{fig:bulk_definition_comparison}.
\begin{align} % Reduction of the plasma dispersion function into a gaussian
	\lim_{b \,\to\, 0} \, \text{Im}\left[X(Z \,+\, b\,i)\right] \,=\, \sqrt{\pi} \,
		\exp(-Z^2) \label{eq:plasma_disp_reduction}
\end{align}
This is reasonable, as $\omega_t$ is consistently two orders of magnitude larger than $\nu_{ii}$.
The complementary error function goes to unity as its real component goes to zero while its imaginary portion is nonzero.
More investigation is needed into the robustness of this form of the bulk viscosity, as well as a comparison with Eq.~\ref{eq:shaing_bulk}.

With this modification, the final form used for this investigation is
\begin{align} % Final form of Bulk Viscosity
	\Gamma_i^{\pi\parallel} \,=\, \,n_e \, \sqrt{\pi} \, D_{\pi\parallel}
		\left(\frac{n^\prime}{n} + \frac{Z}{\rho_{\theta i}}\right) \,
		\exp\left(-Z^2\right) \label{eq:Gamma_bulk_final}
\end{align}

In addition, although this flux, at first glance, may take a similar shape to subsequently-presented fluxes, the plasma dispersion function does not allow for a direct comparison in the form of Eq.~\ref{eq:Gamma_an_g}.


\subsection{Electron Anomalous Diffusion}\label{ssec:an_diffusion}
When a drift wave turbulence interacts with the edge of plasma, it preferentially removes electron momentum out of the confined plasma \cite{itoh_model_1988} \cite{stringer_non-ambipolar_1995}.
The particle flux for this effect can be written as
\begin{align} % Gamma_an original and D_an
	\Gamma_e^\text{an} \,=\, -n_e \, &D_\text{an} \left(\frac{n^\prime}{n} \,+\,
		\frac{\alpha_\text{an}\,T_e^\prime}{T_e} \,+\, \frac{e\,E_r}{T_e}\right)
		\label{eq:Gamma_an_orig} \\
	&D_\text{an} \,=\, \frac{\epsilon^2 \sqrt{\pi}}{2 a_m}
		\frac{\rho_{\theta e} \, T_e}{B} \label{eq:D_an}
\end{align}
The numerical constant $\alpha_\text{an}$ is on the order of unity, and $D_\text{an}$ is the diffusion coefficient for this process.
Note that the electron poloidal gyroradius $\rho_{\theta e}$ is used, which differs only by mass in its definition.
We can normalize the electric field, rewrite the terms with `gradient' coefficients $g_l^\text{k}$ for coefficient $l$ and mechanism $\text{k}$.
\begin{align} % Gamma_an g's
	e\,\Gamma_e^\text{an} \,&=\, g_n^\text{an}\,\frac{n^\prime}{n} \,+\,
		g_T^\text{an}\,\frac{T^\prime}{T} \,+\,
		g_Z^\text{an}\,Z \label{eq:Gamma_an_g} \\
	g_n^\text{an} \,=\, -e \, n \, &D_\text{an}~,~~~~
		g_T^\text{an} \,=\, \alpha_\text{an} \, g_n^\text{an}~,~~~~
		g_Z^\text{an} \,=\, \frac{g_n^\text{an}}{\rho_{\theta i}}
		\label{eq:g_an}
\end{align}


\subsection{Charge Exchange Friction}\label{ssec:cx_friction}
With toroidal rotation, the charge exchange process between ions and neutrals causes a momentum imbalance for the charged plasma, as the momentum goes to the neutrals.
This momentum loss causes a charge imbalance, leading to a current \cite{toda_theoretical_1997}.

One form of the charge exchange rate is obtained from a qualitative scheme by Connor and Wilson \cite{connor_review_2000}, with a ``weak function'' $\phi_\text{cx}$.
\begin{align} % Connor and Wilson's charge exchange rate
	\langle \sigma_\text{cx} v\rangle \,=\, \frac{C_\text{cx}}{\sqrt{m_i}} \,
		\phi_\text{cx}\left(\frac{E_0}{T}\right) \label{eq:connor_cx_rate}
\end{align}
Another form, by Itoh, \emph{et. al} \cite{itoh_model_1989}, follows this idea by making it proportional to the cube root of the temperature.
\begin{align} % Itoh's charge exchange rate
	\langle \sigma_\text{cx} v\rangle \,=\, 10^{-14}
		\left(100 \, T\right)^{1/3} \label{eq:itoh_cx_rate}
\end{align}
We can equate these two, assuming the weak function $\phi_\text{cx}(z) \propto z^{-1/3}$.
\begin{align}
	\frac{C_\text{cx}}{\sqrt{m_i}} \, \left(\frac{E_0}{\cancel{T}}\right)^{-1/3}
		\,=\, 10^{-14} \, (100)^{1/3} \, \cancel{T^{1/3}} \\
	C_\text{cx} \,=\, \frac{4.642\times 10^{-14} \sqrt{m_i}}{\sqrt[3]{E_0}}
		\,\approx\, 7.953\times 10^{-28}
\end{align}
Electrons are not exchanged above a particular temperature threshold, but rather fully ionized and lost to the bulk plasma.
The term $E_0$ is simply the ionization energy for the particular species used; hydrogen is investigated in this project, and the term is thus set to 13.6 eV.
The coefficient $C_\text{cx}$ is a numerical constant that chooses the maximum rate.

In the domain of the problem, the origin of neutrals is that of recycled particles from the divertor, and not that of the NBI.
The profile of these therefore have a penetration depth from the edge, and drops off rapidly.
\begin{align} % Neutrals profile
	n_0 \,\propto\, \frac{\theta\,\Gamma_c}{v_{T_i}\left[1 \,+\,
	\exp{(k(x - d))}\right]} \label{eq:neutral_density}
\end{align}
The maximum value of the neutrals is governed by $\frac{\theta\,\Gamma_c}{v_{T_i}}$, in which $\theta$ is some small value less than unity.
The penetration depth is $d$, with the rate of drop off determined by $k$.

Combining the neutrals density and charge exchange rate into a nonambipolar flux gives
\begin{align} % Charge exchange current
	e\,\Gamma_e^\text{cx} \,&=\,
		-\frac{m_i \,n_0 \langle\sigma_\text{cx} v\rangle \, n\,T}{B_\theta^2}
		\, \left[\frac{B_\theta^2}{\epsilon^2 B_\phi^2} + 2\right] \,
		\left(\frac{n^\prime}{n} \,+\, \frac{\alpha_\text{cx}\,T^\prime}
		{T} + \frac{Z}{\rho_{\theta i}}\right)~. \label{eq:Gamma_cx}
\end{align}
Similar to $\alpha_\text{an}$, the coefficient $\alpha_\text{cx}$ is on the order unity.
We can write the terms in a similar fashion to Eqs.~\ref{eq:Gamma_an_g} as comparison.
\begin{align} % Charge exchange g's
	g_n^\text{cx} \,=\, -\frac{m_i \,n_0 \langle\sigma_\text{cx} v\rangle \, n \, T}
		{B_\theta^2}& \left[\frac{B_\theta^2}{\epsilon^2 B_\phi^2} + 2\right]
		~,~~~~ g_T^\text{cx} \,=\, \alpha^\text{cx}\,g_n^\text{cx}~,~~~~
		g_Z^\text{cx} \,=\, \frac{g_n^\text{cx}}{\rho_{\theta i}}
		\label{eq:g_cx}
\end{align}


\subsection{Ion Orbit Loss}\label{ssec:ol_loss}
The loss of ions due to each individual orbit is a radial current that has certainly been seen in experiments \cite{weisen_boundary_1991}.
Again, due to the fact that ions are significantly more massive than electrons, the size of their gyroradius and banana orbits are significantly different.
When an ion follows a field line which is less than the distance of one gyroradius away from the last close flux surface, the ion is lost.
This current, also referred to as the ion loss cone, is compounded by the fact that it is highly affected by the electric field.
\begin{align} % Ion orbit loss definition
	\Gamma_i^\text{ol} \,=\, n \, \rho_{\theta i} \, \nu_{ii} \, \nu_{*i} \,
		\frac{\exp\left[-\sqrt{\nu_{*i} + Z^4
		+ \left(\frac{x}{w_{bi}}\right)^4}\right]}{\sqrt{\nu_{*i} + Z^4
		+ \left(\frac{x}{w_{bi}}\right)^4}} \label{eq:Gamma_ol}
\end{align}
The term $\nu_{*i}$ is the ion-ion collision frequency upon the ion banana bounce frequency. %, and $\nu_{in_0}$ is the ion-neutral collision frequency.
%The sum of the ion-ion and ion-neutral frequency serves as the effective detrapping frequency.
This particular form highly localizes the effect of the radial electric field \cite{kobayashi_experimental_2016}.

Similar to bulk viscosity, and contrasting charge exchange friction and electron anomalous diffusion, this flux is explicitly nonlinear in $Z$.


\section{Radial Electric Field Equation}\label{sec:Z_equation}
A rigorous theory on what causes the diffusion coefficients $D$ and $\chi$ is still unknown.
%However, experiments and mathematical investigations indicate that flows in the plasma can tear apart turbulent eddies, which reduces the transport radially.\todo{\color{red}{Reword?}}
To model the dynamics of the transition fully within this scope, the radial electric field must also be evolved explicitly.
Two forms of the field equation are presented, with the latter being of more interest.

\subsection{Original Model}\label{ssec:original_Z_equation}
Itoh \emph{et al.} \cite{itoh_edge_1991} and Zohm \cite{zohm_dynamic_1994} originally modeled an evolution to the radial electric field, which included hysteresis and an oscillatory phase.
It was incorporated from the FitzHugh-Nagumo model, Eq.~\ref{eq:FitzHugh-Nagumo}.
\begin{align} % Original Itoh model
	\frac{\partial Z}{\partial t} \,&=\, -N(Z,g) \,+\, \mu
		\frac{\partial Z^2}{\partial x^2}~,\label{eq:original_z} \\
	N(Z,g) \,&\equiv\, g \,-\, g_0 \,+\, \left[\beta Z^3 \,-\, \alpha Z\right]~.
\end{align}
The nonlinear $N(Z,g)$ function introduces the bifurcating behavior, with $g$ acting as a gradient parameter, and $g_0$, $\alpha$, and $\beta$ as the bifurcation parameters \cite{itoh_model_1988}.
Weymiens \cite{weymiens_bifurcation_2012} rewrote the equation as
\begin{align} % Weymiens ('original') model
	\epsilon \, \frac{\partial Z}{\partial t} \,=\, \mu \,
		\frac{\partial^2 Z}{\partial x^2} \,+\, \frac{c_n T}{n^2} \,
		\frac{\partial n}{\partial x} \,+\, \frac{c_T}{n} \,
		\frac{\partial T}{\partial x} \,+\, G(Z)~,\label{eq:original_Z_equation} \\
	G(Z) \,\equiv\, a \,+\, b(Z - Z_S) \,+\, c(Z - Z_S)^3~.
		\label{eq:G_polynomial}
\end{align}
This form has includes the polarization current on the left-hand side and the shear viscosity as the first term on the right-hand side.
All other possible sources of current are consolidated as the remaining terms, bundling the density and temperature gradient terms accordingly.
The polynomial $G(Z)$ is cubic with an inflection point at $Z_S$, giving the transition behavior.
The type of transition is wholly dictated by the values of the coefficients $a$, $b$, and $c$.

The coefficients that are chosen for demonstration are the following.
The particle-thermal diffusivity coupling constant $\zeta$ and the dynamic viscosity $\mu$ are used both in the original and flux models.
\begin{table}[h] % Chosen parameters
\centering
\begin{tabular}{cc|ccccccc}
	$\zeta$ & $\mu$ & $\epsilon$ & $c_n$ & $c_T$ & $a$ & $b$ & $c$ & $Z_S$ \\ \hline
	0.5 & $\frac{1}{20}$ & $\frac{1}{25}$ & -1.1 & -0.9 & 1.5 & 2.0 & -1.0 & -1.5
\end{tabular}
\end{table}

\subsection{Flux Model}\label{ssec:flux_Z_equation}
Rather than qualitatively creating an equation as the original model, a synthesis of the nonambipolar fluxes from Section~\ref{sec:nonambipolar_fluxes} is substituted into Eq.~\ref{eq:ambipolarity_constraint}.
\begin{align} % Sum of the currents
	\epsilon_0 \, \frac{\partial E_r}{\partial x} \,&=\, -J^\text{pol}
		- J^{\pi\perp} - J^{\pi\parallel} + J^\text{an} - J^\text{cx}
		- J^\text{ol} \label{eq:current_sum}
%	&=\, -\frac{e \, n \, \rho_{\theta i}}{2} \, \frac{\partial Z}{\partial t}
%		\,+\, \frac{e \, \mu \, \rho_{\theta i}}{2} \, \frac{\partial}
%		{\partial x} \left[n \, \frac{\partial Z}{\partial x}\right]
%		\,-\, n_e \, D_{\pi\parallel} \left(\frac{n^\prime}{n} + \frac{Z}
%		{\rho_{\theta i}}\right) \, \text{Im}\left[X\left(Z
%		\,+\, i\,\frac{\nu_{ii}}{\omega_t}\right)\right] \\ &\,+\,
%		-n_e \, D^\text{an} \left(\frac{n^\prime}{n} \,+\,
%		\frac{\alpha_\text{an}\,T_e^\prime}{T_e} \,+\,
%	\frac{e\,E_r}{T_e}\right) \\ &\,-\, -\frac{m_i \,n_0
%		\langle\sigma_\text{cx} v\rangle \, n\,T}{B_\theta^2}
%		\, \left[\frac{B_\theta^2}{\epsilon^2 B_\phi^2} + 2\right] \,
%		\left(\frac{n^\prime}{n} \,+\, \frac{\alpha_\text{cx}\,T^\prime}
%		{T} - \frac{Z}{\rho_{\theta i}}\right) \\
%		&\,-\, n \, \rho_{\theta i} \, \nu_{ii} \, \nu_{*i} \,
%		\frac{\exp\left[-\sqrt{\nu_{*i} + Z^4
%		+ \left(\frac{x}{w_{bi}}\right)^4}\right]}{\sqrt{\nu_{*i} + Z^4
%		+ \left(\frac{x}{w_{bi}}\right)^4}}
\end{align}
It is advantageous to bundle the neoclassical polarization current $J^\text{pol}$ with the left-hand side before it is normalized, \emph{i.e.} Eq.~\ref{eq:polarization_current_original}.
\begin{align} % Non-normalized current equation
	\left(\epsilon_0 \,+\, \frac{m_i \, n}{B_\theta^2}\right) \,
	\frac{\partial E_r}{\partial x} \,&=\, -J^{\pi\perp} \,-\, J^{\pi\parallel}
		\,+\, J^\text{an} \,-\, J^\text{cx} - J^\text{ol}
\end{align}
With typical plasma parameters at the edge, \emph{i.e.} $n \sim 10^{19}~\text{m}^{-3}$, $B_\theta \sim 10^{-1}$, the coefficient in $J^\text{pol}$ is 4 orders of magnitude larger than $\epsilon_0$, and therefore dominates.
After normalizing this and the shear viscosity, the balance of currents is
\begin{align} % Currents equation
	\frac{e \, n \, \rho_{\theta i}}{2} \, \frac{\partial Z}{\partial t}
		\,=\, \frac{e \, \rho_{\theta i}}{2} \frac{\partial}{\partial x}
		\left[\mu \, n \, \frac{\partial Z}{\partial x}\right] \,-\,
		J^{\pi\parallel} \,+\, J^\text{an} \,-\, J^\text{cx} - J^\text{ol}
		\label{eq:normalized_Z_equation}
\end{align}
A further simplification can be done by multiplying everything by $2 / (e \, \rho_{\theta i})$.
\begin{align} % Final Equation!
	n \, \frac{\partial Z}{\partial t} \,=\, \frac{\partial}{\partial x}
		\left[\mu \, n \, \frac{\partial Z}{\partial x}\right] \,+\,
		\frac{2}{\rho_{\theta i}} \left(-\Gamma^{\pi\parallel} \,+\,
		\Gamma^\text{an} \,-\, \Gamma^\text{cx} \,-\, \Gamma^\text{ol}\right)
		\label{eq:reduced_normalized_Z_equation}
\end{align}

\section{Forms of Diffusivity}\label{sec:diffusivity}
Various forms of $D$ (and subsequently $\chi$) are presented in investigations with this model.
The original Itoh model has $D$ as a sigmoid function of the normalized radial electric field.
Specifically, it was proportional to the hyperbolic tangent of the field, with a minimum and maximum diffusivity \cite{itoh_edge_1991, zohm_dynamic_1994}
\begin{align} % Itoh diffusivity
	D(Z) \,&=\, \frac{D_\text{max} + D_\text{min}}{2} \,+\,
		\frac{(D_\text{max} - D_\text{min})\tanh(Z)}{2}~.
		\label{eq:Itoh_diffusivity}
\end{align}

However, it is now accepted that the diffusivity must be a function of the shear of the field.
This is either due to the stabilization of linear modes, or through the reduction of turbulence amplitudes, correlation lengths, or change in phases of the fluctuations \cite{connor_review_2000}.
It was found that ``a flattened (steep) radial equilibrium gradient tends to enhance (eliminate) turbulence suppression due to the shear flow'' \cite{zhang_edge_1992}.
\begin{align} % Shear diffusivity
	D(Z^{\prime}) \,=\, D_\text{min} \,+\, \frac{D_\text{max} - D_\text{min}}
		{1 \,+\, c(Z^{\prime})^{\beta}} \label{eq:shear_diffusivity}
\end{align}
Previous investigations using this form have had a variety of choices for the parameters, including $c = 0.25, \beta = 2$, and $c = 1, \beta = 2/3$ \cite{connor_review_2000} \cite{itoh_theoretical_1994}.
In the cases of non-integer values of $\beta$, the absolute value of the derivative is used.

Another form includes both the field itself and its shear \cite{paquay_studying_2012}, and will be referred to as the Flow-Shear model.
\begin{align} % Flow-Shear equation
	D(Z, Z^{\prime}) \,=\, D_\text{min} \,+\,
		\frac{D_\text{max} - D_\text{min}}{1 \,+\, a_1 Z^2 \,+\,
		a_2 Z \cdot Z^{\prime} \,+\, a_3 (Z^{\prime})^2}
		\label{eq:flow_shear_diffusivity}
\end{align}
The constant $a_1$ allows for the value of the radial electric field to also contribute to the diffusivity.
This form has no literature support (and physical support, for that matter) for $a_2$, and therefore $a_2$ is set to zero.

\section{Boundary Conditions and Initial Profiles}\label{sec:boundary_conditions}
The set of aforementioned PDE's is evaluated in a domain that is small enough to exclude the inner (core) boundary from any possible particle and heat sources.
In this model, this boundary to the domain is $x = L$, for some positive length $L$, with the core is assumed to be at $x \gg L$.
It also requires the domain to be sufficiently large to fully-encompass the transport barrier and ignore any possible boundary effects of the barrier itself.
The outer boundary of the plasma edge is assumed to be the start of the SOL (scrape-off layer); this corresponds to $x = 0$.

It is assumed that the SOL is passive to particle and heat fluxes, and therefore the density and temperature profiles exponentially decay.
\begin{align} % n and T boundary conditions at SOL
	n^\prime(0) \,=\, \frac{n}{\lambda_n}~, ~~~~
		T^\prime(0) \,=\, \frac{T}{\lambda_T}~. \label{eq:SOL_boundary}
\end{align}
There are hypothesized boundary conditions for the electric field at the edge, but have not been confirmed.
The system does, however, still exhibit bifurcating behavior (dithering, hysteresis) with this boundary condition \cite{paquay_studying_2012}.
Two of these are the following:
\begin{align} % Edge Z conditions
	Z^\prime(0) \,=\, \frac{Z}{\lambda_Z} ~~~ \text{or} ~~~ Z^\prime(0) \,=\,
		0 \label{eq:Z_edge_conds}
\end{align}

The typical decay length scales are chosen as the following, with ``Original'' referring to the system solving Eq.~\ref{eq:original_Z_equation}, and ``Flux'' referring to the system solving Eq.~\ref{eq:reduced_normalized_Z_equation}.
\begin{table}[h] % Decay lengths at the edge
\centering
\begin{tabular}{r|ccc}
	& $\lambda_n$ & $\lambda_T$ & $\lambda_Z$ \\ \hline
	Original & 1.25 & 1.5 & 1.25 \\ \hline
	Flux & 0.01 & 0.0125 & 0.01
\end{tabular}
\end{table}

The core side has a constant influx of particles and heat, stemming from fueling and heating.
Using Eqs.~\ref{eq:particle_flux} and \ref{eq:heat_flux}, we can arrive at Neumann- and Robin-type boundary conditions, respectively.
\begin{align} % Fluxes from core
	\Gamma(L) \,&=\, \Gamma_c ~\longrightarrow~ n^\prime(L)
		\,=\, \frac{\Gamma_c}{D} \label{eq:core_particle_flux}\\
	q(L) \,&=\, q_c ~\longrightarrow~ T^\prime(L) \,=\, \frac{\zeta(
		\Gamma_c \, T - (\gamma - 1)\,q_c)}{D \, n} \label{eq:core_heat_flux}
\end{align}
The electric field shear must vanish as it approaches the core.
\begin{align} % Electric field towards the core
	Z^\prime(L) \,=\, 0 \label{eq:core_Z_boundary}
\end{align}

The initial conditions of this system are not one of huge importance if chosen reasonably.
Nevertheless, for consistency, the conditions put forth by Paquay \cite{paquay_studying_2012} are used.
They are derived for when $Z$ is a constant and in steady-state.
\begin{align} % Paquay's initial conditions
	n(x,0) \,&=\, -\frac{\Gamma_c \, \lambda_n}{D} \left(1 \,+\,
		\frac{x}{\lambda_n}\right) \label{eq:n_initial} \\
	T(x,0) \,&=\, \frac{q_c (\gamma \,-\, 1)}{\Gamma_c} \left[1 \,-\,
		\frac{\lambda_n}{\zeta\lambda_T \,+\, \lambda_n} \left(1 \,+\,
		\frac{x}{\lambda_n}\right)^{-\zeta}\right]\label{eq:T_initial}
\end{align}
One could also conceivably choose simple linear profiles, as the system would go into its steady-state relatively quickly.

Together with Appendix~\ref{chapter:Plasma_Parameters}, the model is ready to be implemented in one's PDE solver of choice.
The following chapter describes the choice made for this investigation using FiPy \cite{guyer_fipy:_2009}, a solver written in Python  that uses the Finite Volume Method.

