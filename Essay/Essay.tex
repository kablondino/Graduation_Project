\documentclass[12pt,a4paper]{letter}
\name{Kevin A. Blondino}
\signature{Kevin A. Blondino}
\date{\today}

\address{
	Flux, Groene Loper 19,\\
	5612 AZ Eindhoven
}

\usepackage{fullpage, indentfirst, amsmath}
\linespread{1.2}
\usepackage[UKenglish]{isodate}
\cleanlookdateon
\usepackage[T1]{fontenc}
\usepackage[utf8]{inputenc}

% Temporary Packages
\usepackage{lipsum, color}

\begin{document}
\begin{letter}{
	Bloomingdale Senior High School,\\
	1700 E Bloomingdale Ave,\\
	Valrico, FL 33596, USA
}

\opening{To the science students of Bloomingdale,}

The term ``holy grail'' is usually used to indicate some physical object or abstract goal with great significance or utility.
Sometimes, it also implies it is elusive or somehow beyond reach.
One may argue that for the energy industry, nuclear fusion is considered its holy grail.
{\color{red}FINISH INTRO!}

%Any origin story of nuclear energy must begin with Einstein's famous discovery, and probably the single most famous equation of all time, of $E = m\,c^2$.
%It conveys a powerful, but still simple, statement about the fundamental rules of the universe.
%Its technical name gives the best description: the mass-energy equivalence.
%Mass and energy are not separate things, but different manifestations of the same thing.
%The factor of $c^2$ means that a small amount of mass $m$ can ``become'' a whole lot of energy.
%The nucleus of heavy atoms are strangely less massive than the combined sum of the 
The public's perception of nuclear energy is usually one of fear and misunderstanding.
Simply the word ``nuclear'' conjures images of large explosions, deadly radiation, and a nuclear wasteland.
However, current nuclear energy technology has proven to be safer than traditional energy sources; however it still retains serious problems.
There are two ways to tap into the energy stored in the nucleus, with one manner being significantly easier.

Currently, most promising device for the generation of energy from nuclear fusion is called a tokamak.

My research involved investigating the edge of the plasma.

\closing{Sincerely,}
\signature
\name

\end{letter}
\end{document}

