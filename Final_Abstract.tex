
The viability of tokamaks as a energy source relies on its operation mode, as anomalous transport dominates confinement.
H--mode is defined as a reduction in this transport; there does not yet exist a comprehensive theory of the fundamental physics of this mode and the L--H transition.
What is established is that a radial electric field at the plasma edge is known to be an integral part in the processes of this operational mode.

The topic of this study is determine the dominance of mechanisms that generate this radial electric field.
A mathematical model is used to simulate the dynamics near the edge of the plasma, with many of the effects in generating and suppressing the field as additive terms in an equation for a radial displacement current.
The forms of these mechanisms can greatly affect the underlying mathematical behavior.

The dynamics of the system generated both expected and unexpected results.
An operational mode somewhat resembling H--mode was formed when supplied by adequate input power.
One nonambipolar flux was found to dominate: the ion bulk viscosity.
However, unphysical oscillations and occasional lack of a steady-state were found in the system, with no current explanation.
The forms of the mechanisms therefore require more scrutiny.

