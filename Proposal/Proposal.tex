\documentclass{article}
\author{Kevin A. Blondino \\
	\texttt{Supervisor: dr. H.J. de Blank}}
\title{Radial Electric Field-Generating Terms in L-H Transitions}
\date{}

\usepackage{fullpage,amsmath,natbib,graphicx,indentfirst}

\usepackage{lipsum}

\usepackage{xcolor}
\newcommand\mynotes[1]{\textcolor{red}{#1}}

\linespread{1.05}

%--------------------------------------

\begin{document}
\maketitle

%--------------------------------------

\begin{abstract}
	\mynotes{\lipsum[1]}
	From the discovery of H-mode by ASDEX in 1982, inquiry in the transition mechanisms have become a topic of interest.
\end{abstract}

\section{Introduction and Background}
In 1982, a new NBI system was install in the ASDEX tokamak, which pushed into a new realm.
A new level of energy confinement time was achieved, typically a factor of 2 or more than previous.
This state of operation was coined the high-confinement (H-) mode, and is considered necessary for the future of nuclear fusion as an energy source \cite{wagner_development_1984}.

The transport of particles and energy has been discovered to be dominated by anomalous transport, which is generally assumed to be generated by turbulence driven by micro-instabilities \cite{freidberg_plasma_2007}.
The transition between low-confinement (L-) mode and H-mode is a bifurcation in the turbulent transport at the edge of the tokamak in a divertor setup.
A key mechanism for the transition is the suppresion of this transport by the generation of a large radial electric field and the corresponding $\mathbf{E}\times\mathbf{B}$ flows and large flow shear near the edge.
A plethora of individual mechanisms for the generation of such an electric field have been proposed, most of which can be viewed as separate contributions in a radial Poisson's law.

The radial electric field $E_r$ is deduced from the radial force balance for any plasma species $j$, as follows:
\begin{equation}
	E_r \,=\, -\frac{1}{n_j e_j} \frac{\text{d} p_j}{\text{d} r} + V_{\theta j} B_\phi - V_{\phi j} B_\theta
	\label{eq:E_r}
\end{equation}
In the above, $e_j$ represents the charge of the $j$-th species, $n_j$ is the density, $p_j$ is the pressure, and $V_{\theta j}$ and $V_{\phi j}$ are the poloidal and toroidal velocities, respectively.
This grants changes in the radial electric field to be associated with changes in radial gradient of the pressure or either velocities \cite{connor_review_2000}\cite{staps_backstepping_2017}.

The basic model for the transition, developed first by Itoh et al. \cite{itoh_edge_1991} and later expanded by Zohm \cite{zohm_dynamic_1994}, contains the continuity equations of energy and density to describe H-mode.
The generation of a transport barrier could then be caused by a local reduction of the transport coeffecients.
Such a model was further adapted by Weymiens \cite{weymiens_bifurcation_2014} into the following form.
\begin{equation}
	\epsilon \frac{\partial Z}{\partial t} \,=\, \mu \frac{\partial^2 Z}{\partial x^2} + \frac{c_n T}{n^2} \frac{\partial n}{\partial x} + \frac{c_T}{n} \frac{\partial T}{\partial x} + G(Z)
	\label{eq:pde}
\end{equation}
This model uses the $Z$, the radial electric field normalized with respect to the ion Larmor radius $\rho_{pi}$ and temperature $T$.
\begin{equation}
	Z \,=\, \frac{\rho_{pi} e E_r}{T}
	\label{eq:normalization}
\end{equation}
% Another way to look at this equation is that the term on the left is the polarization radial current, the 2nd-derivative is due to shear viscosity.
The left-hand side of Eq.~\ref{eq:pde} describes the radial current due to polarization, with $\epsilon = B_\theta^2 / (B^2 \nu_i)$, the dielectric constant.
The 2nd-derivative term describes the radial current due to the anomalous shear viscosity of the $\mathbf{E}\times\mathbf{B}$ drift, with $\mu$ as the ratio of viscosity to collisition frequency.
The 1st-derivative terms are due to all ambipolar cross-field fluxes.
$G(Z)$ is a cubic polynomial that approximates the remaining electric field-inducing contributions.
It has an inflection point at in $Z$-space at $Z_S$ and coefficients $a$, $b$, and $c$:
\begin{equation}
	G(Z) \,=\, a + b(Z - Z_S) + c(Z - Z_S)^3
\end{equation}

%--------------------------------------

\section{Problem Statement}


%--------------------------------------

\section{Project Plan}


%--------------------------------------

% References
%\newpage
\bibliographystyle{plain}
%\nocite{*}
\bibliography{../References/References}
\end{document}
