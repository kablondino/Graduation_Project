\documentclass{article}
\author{Kevin A. Blondino \\
	Supervisor: Dr. H.J. de Blank}
\title{Radial Electric Field-Generating Terms in L-H Transitions}
\date{}

\usepackage{fullpage,amsmath,natbib,graphicx,indentfirst}

\usepackage{lipsum}

\usepackage{xcolor}
\newcommand\mynotes[1]{\textcolor{red}{#1}}

%--------------------------------------

\begin{document}
\maketitle

%--------------------------------------

\begin{abstract}
	\mynotes{\lipsum[1]}
	From the discovery of H-mode by ASDEX in 1982, inquiry of the transition mechanisms have become a topic of interest.
\end{abstract}

\section{Introduction and Background}
In 1982, a new NBI system was install in the ASDEX tokamak, which pushed into a new realm.
A new level of energy confinement time was achieved, typically a factor of 2 or more than previous.
This state of operation was coined the high confinement (H-) mode, and is considered necessary for the future of nuclear fusion as an energy source \cite{wagner_development_1984}.

The transport of particles and energy are dominated by turbulence \mynotes{CITE?}.
The transition between low-confinement (L-) mode and H-mode is a bifurcation in the turbulent transport at the edge of the tokamak in a divertor setup.
A key mechanism for the transition is the suppresion of this transport by the generation of a large radial electric field and the corresponding $\mathbf{E}\times\mathbf{B}$ flows and large flow shear near the edge.
A plethora of individual mechanisms for the generation of such an electric field have been proposed, as most of which can be viewd as separate contributions in a radial Poisson's law.

%--------------------------------------

\section{Problem Statement}


%--------------------------------------

\section{Project Plan}


%--------------------------------------

% References
%\newpage
\bibliographystyle{plain}
%\nocite{*}
\bibliography{../References/References}
\end{document}
